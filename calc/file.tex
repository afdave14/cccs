\documentclass[12pt]{article}
\usepackage{amsthm,amssymb,amsmath}

\newcommand{\piRsquare}{\pi r^2}		% This is my own macro !!!

\title{\textbf{Chain Rule and proof}}			% used by \maketitle
\author{David Campbell, }		% used by \maketitle
\date{8 September, 2012}					% used by \maketitle

\numberwithin{equation}{section}

\theoremstyle{definition}
\newtheorem*{chrule}{Chain Rule}

\begin{document}
\maketitle						% automatic title!
\begin{chrule}
\end{chrule}

Assume:
\begin{equation*}
f(x)=g(h(x))
\end{equation*}

Then:
\begin{equation*}
f\prime(x) = g\prime(x)\cdot{h}\prime(g(x))
\end{equation*}

\begin{proof}
\begin{multline*}
\mbox{   Let}
\end{multline*}
\begin{equation*}
y=f(g(x))
\end{equation*}
\indent{so,}
\begin{equation*}
\frac{dy}{dx}=\lim_{h \to 0}\frac{f(g(x+h)-f(g(x)))}{h}
\end{equation*}
\newline\newline
\indent{Now let}
\begin{equation*}
k=g(x+h)-g(x)\Rightarrow
g(x+h)=k+g(x)
\end{equation*}
\indent{then}
\begin{equation*}
\frac{f(g(x+h))-f(g(x))}{h}\Rightarrow
\frac{f(g(x)+k)-f(g(x))}{h}\Rightarrow
\end{equation*}
\begin{equation*}
\frac{f(g(x)+k)-f(g(x))}{k}\cdot\frac{k}{h}\Rightarrow
\frac{f(g(x)+k)-f(g(x))}{k}\cdot\frac{g(x+h)-g(x)}{h}
\end{equation*}
\newline\newline
\indent{Now let}
\begin{equation*}
u=g(x)
\end{equation*}
\indent{so now}
\begin{equation*}
\frac{f(u+k)-f(u)}{k}\cdot\frac{g(x+h)-u}{h}
\end{equation*}
\newline\newline
\indent{Since g is differentiable at x, it is continuous there:}
\newline
\begin{equation*}as\mbox{ h}\rightarrow
0,\mbox{ g}(x+h)\rightarrow(x)\mbox{ and we must have }k\rightarrow{0}\end{equation*}
\newline\newline
\indent{That is}
\begin{equation*}\lim_{h \to 0}\frac{f(g(x+h))-f(g(x))}{h}\end{equation*}
\begin{equation*}=\lim_{k \to 0}\frac{f(u+k)-f(u)}{k}\cdot\lim_{h \to 0}\frac{g(x+h)-g(x)}{h}\end{equation*}
\newline\newline
\indent{So, since}
\begin{equation*}u=g(x)\mbox{ we have }\frac{df(g(x))}{dg(x)}\cdot\frac{d(g(x))}{dx}\end{equation*}
\indent{which is the Chain Rule.}
\end{proof}

\end{document}             % End of document.