\input{myassignmentpreamble}
\input{yliow}
\input{ciss362}

\renewcommand\TITLE{Test 1 Part B}
\renewcommand\AUTHOR{David Campbell}
\renewcommand\EMAIL{djcampbell2@cougars.ccis.edu}

\begin{document}
\topmatter


We fix an alphabet $\Sigma$ throughout.
The following are facts you have already seen.

{\bf Concatenation.}
For the following $x,y,z$ are words in $\Sigma^*$.
\begin{enumerate}
\item[C1] $xy$ is a word in $\Sigma^*$
\item[C2] $(xy)z$ can be rewritten as $x(yz)$
\item[C3] $x(yz)$ can be rewritten as $(xy)z$
\item[C4] $\ep x$ can be rewritten as $x$
\item[C5] $x$ can be rewritten as $\ep x$
\item[C6] $x\ep$ can be rewritten as $x$
\item[C7] $x$ can be rewritten as $x\ep$
\item[C8] If $x \in \Sigma^*$ is not $\ep$, then 
$x = x'x''$ for some $x' \in \Sigma$ and $x'' \in \Sigma^*$.
\end{enumerate}

{\bf Length function.}
Let $x$ be a word over $\Sigma^*$.
\begin{enumerate}
\item[L1] If $x = \ep$, then $|x| = 0$.
\item[L2] If $x \neq \ep$, then $x = x' \cdot x''$ where
$x \in \Sigma$ (i.e. $x$ is a symbol in our alphabet)
and $x'' \in \Sigma^*$ (i.e. $x''$ is a word over $\Sigma$).
Then $|x| = 1 + |x''|$.
\item[L3] $|x| \geq 0$
\item[L4] $|x|$ is an integer 
\item[L5] If $|x| = 0$, then $x = \ep$.
\item[L6] $|x| \neq 0 \iff x \neq \ep$
\end{enumerate}
L5-L6 were proven in an earlier assignment.
L6 is essentially L1 and L5.


{\bf Reverse function.} 
Let $x \in \Sigma^*$.
\begin{enumerate}
\item[R1] If $x = \ep$, then $\ep^R = \ep$.
\item[R2] If $x \neq \ep$, then $x = x'\cdot x''$ where
$x' \in \Sigma$ ($x'$ is a symbol in our alphabet $\Sigma$) 
and $x'' \in \Sigma^*$ (i.e., $x''$ is a word over $\Sigma$) 
and $x^R = (x'')^R \cdot x'$.
\item[R3] If $x,y$ are words in $\Sigma^*$, then
$(xy)^R = y^R x^R$
\end{enumerate}
R3 was proven in an earlier assignment.

\newpage

Read this document very carefully.
I'll be using very formal notation for DFA.
Remember that I prefer to write the formal definition of a DFA
as $(\Sigma, Q, q_0, \delta, F)$.
This is slightly different from your textbook which begins with 
$Q$ instead. 

The goal of this assignment is to formalize acceptance.
Our definition of DFA acceptance, i.e., 
\lq\lq start at $q_0$, read one character at a time left-to-right,
following the transition edge; if you stop at an accept state then
the word is accepted, otherwise it's not''
is correct but not precise enough for a clear proof.

We will not formalize the above definition.

First of all the action of \lq\lq follow the transitions''
is done using the edges (pictorially) or using the transition
function. 
The fact
\[
\text{\lq\lq at state $q$ if we read symbol $a$ we go to $q'$''}
\]
is the same as
\[
\delta(q, a) = q'
\] 
The two are the same.
However the transition function is better suited for a clear
and formally correct proof.

The transition function only describes one step in the computation.
Here \lq\lq computation'' means \lq\lq go from one state to another''.
Acceptance is carrying out a sequence of computations.
So the first thing is to extend the concept of one computation
to a sequence to computation until all symbols in a string is 
completely processed.
We will therefore define a function
\[
\delta^*
\]
(In some books this is denoted $\widehat{\delta}$.)
Basically this is what we want $\delta^*$ to do:
Suppose a $\delta$ function does this for a word $aba$:
\begin{align*}
\delta(q_0, a) &= q_5 \\
\delta(q_5, b) &= q_3 \\
\delta(q_3, a) &= q_2 \\
\end{align*}
and $q_2$ is an accept state, then we accept $aba$.
Pictorially this means
\begin{center}
\begin{tikzpicture}[shorten >=1pt,node distance=2cm,auto,initial text=]
\node[state,initial]   (q_0)          {$q_0$};
\node[state]           (q_5) at (2,0) {$q_5$};
\node[state]           (q_3) at (4,0) {$q_3$};
\node[state,accepting] (q_2) at (6,0) {$q_2$};

\path[->] (q_0) edge node {$a$} (q_5)
          (q_5) edge node {$b$} (q_3)
          (q_3) edge node {$a$} (q_2)
;
\end{tikzpicture}
\end{center}
(other parts of the DFA diagram is of course not shown.)
If that's the case, then for this $\delta$ we want the $\delta^*$ to 
do this
\[
\delta^*(q_0, aba) = q_2
\]
In other words, $\delta^*$ basically computes the resulting state
after consuming all the symbols in the word.

Obviously given a $\delta$, the definition of $\delta^*$ depends on $\delta$.
Specifically, we will define it recursively.
First if there's nothing to read of course we have
\[
\delta^*(q, \ep) = q
\]
for every state $q$ in the DFA.
We have already handled the case where $|x| = 0$.
Therefore we only need to look at the case where $|x| > 0$.
In that case $x = x'x''$ where $x'$ is a symbol and $x''$ is a word.
To model the behavior of 
\lq\lq processing one symbol at a time going left-to-right
and following $\delta$'', the only reasonble definition of $\delta^*$ is
\[
\delta^*(q, x'x'') = \delta^* \biggl( \delta(q,x'), \,\,\, x''\biggr)
\]
[If there's exactly one symbol to process, of course $\delta^*$ is exactly
the same as $\delta$:
\[
\delta^*(q, c) = \delta(q, c)
\]
for any state $q$ and any symbol $c$ in the alphabet of the DFA.
You will be proving this later.]
Finally we want to define
\[
\delta^*(q, x)
\]
where $x$ is a a word.

Make sure you really understand the above definition!

In summary, given a function
\[
\delta : Q \times \Sigma \rightarrow Q
\]
we define the corresponding $\delta^*$ to be the function
\[
\delta^* : Q \times \Sigma^* \rightarrow Q
\]
by
\[
\delta^*(q, x) = 
\begin{cases}
q            & \text{ if $x = \ep$} \\
\delta^* ( \delta(q,x'), x'')
             & \text{ if } x = x'x'',x' \in \Sigma, x'' \in \Sigma^* \\
\end{cases}
\]
The function $\delta^*$ is defined recursively in terms of $\delta$.

Let me state formally the properties of $\delta^*$ for the sake of your
proof writing:

Let $\delta : Q \times \Sigma \rightarrow Q$ be a function.
We define an associated function
\[
\delta^* : Q \times \Sigma^* \rightarrow Q
\]
recursively so that:
\begin{enumerate}
\item[DS1] $\delta^*(q, \ep)$ can be rewritten as $q$.
\item[DS2] $q$ can be rewritten as $\delta^*(q, \ep)$.
\item[DS3] If $x' \in \Sigma$ and $x'' \in \Sigma^*$, then
            $\delta^*(q, x'x'')$ can be rewritten as 
          $\delta^*(\delta(q, x'), x'')$.
\item[DS4] If $x' \in \Sigma$ and $x'' \in \Sigma^*$, then
          $\delta^*(\delta(q, x'), x'')$
          can be rewritten as 
          $\delta^*(q, x'x'')$.
\end{enumerate}

Your are strongly advised to test your understanding of the above
definition of $\delta^*$ with the following example:

You are given the following DFA $M$ where
$\Sigma = \{a, b\}$, $Q = \{q_0, q_1, q_2, q_3\}$, the initial state is $q_0$,
$F = \{q_0, q_1, q_2\}$, and $\delta$ is given by
\begin{align*}
\delta(q_0, a) &= q_0 \\
\delta(q_0, b) &= q_1 \\
\delta(q_1, a) &= q_0 \\
\delta(q_1, b) &= q_2 \\
\delta(q_2, a) &= q_0 \\
\delta(q_2, b) &= q_3 \\
\delta(q_3, a) &= q_3 \\
\delta(q_3, b) &= q_3 \\
\end{align*}
Compute $\delta^*(q_0, abab)$, $\delta^*(q_1, abbb)$ using DS1-DS4
and of course
$\delta$, listing down carefully which of the above DS1-DS4 you are using for
each step of the computation.
When you're done, draw the DFA and check that your computation is correct.

Again, just like the case of the reverse function and the length function,
you can very quickly write a $\delta^*$ function given a $\delta$.
For instance in C++, if states are represented using integers, it might
look like this where 
verb!delta! is a transition function already defined:
\begin{Verbatim}[frame=single]
int delta(int q, char c)
{
    ...
}

int deltastar(int q, const char s[])
{
    if (strlen(s) == 0) 
    {
        return q;
    }
    else 
    {
        return deltastar(delta(q, s[0]), s + 1);
    }
}
\end{Verbatim}

\newpage
{\bf Acceptance}

With this precise definition of $\delta^*$ that computes the resulting
state for a word (and not just a symbol), we can formalize the concept of 
acceptance: A word $x \in \Sigma^*$ is accepted by a DFA 
$M = (\Sigma, Q, q_0, \delta, F)$ if
\[
\delta^*(q_0, x) \in F
\]
This acceptance criteria is defined in terms of our $\delta^*$
which depends on $\delta$.
This is now formally our definiton of acceptance.
From this we can quickly define the concept of the language accepted
by our DFA $M$:
\[
L(M) = \{x \in \Sigma^* \mid \delta^*(q_0, x) \in F \}
\]
Let's list the following for proof writing.
Given a DFA $M = (\Sigma, Q, q_0, \delta, F)$, we have:
\begin{enumerate}
\item[AC1] \lq\lq $x$ is accepted by $M$''
          can be replaced by $\delta^*(q_0, \ep) \in F$.
\item[AC2] $\delta^*(q_0, \ep) \in F$
          can be replaced by
          \lq\lq $x$ is accepted by $M$''
\item[AC3] $L(M)$ can be replaced by $\{x \mid \delta^*(q_0, x) \in F \}$
\item[AC4] $\{x \mid \delta^*(q_0, x) \in F \}$ can be replaced by $L(M)$
\end{enumerate}




\newpage
Here's a warmup:

Q1. Let $(\Sigma, Q, q_0, \delta, F)$ be a DFA and $q \in Q$ be fixed. 
Prove that if $x,y \in \Sigma^*$, then
\[
\delta^*(q, xy) = \delta^*(\delta^*(q, x), y)
\]
[Make sure you state the $P(n)$, the smallest value of $n$ you need
for $P(n)$ to hold.
Make sure you prove the base case.
Make sure you prove the inductive case.
Make sure your proof is clear, concise, correct and refers to 
only basic axioms/properties listed.
Follow the format of previous proof problems.
Proper math writing includes proper writing in general.

If you need some fact but can't prove it, you might ask if I'm
willing to include it as a list of facts that you can quote.
That's not a promise that I'll say yes.
]

\SOLUTION
\begin{align*}
           1 a &= a                                     & & \text{by M4} \\
\THEREFORE 1 c &= a \text{ for some integer $c (= a)$ } & & \\
\THEREFORE 1   &\mid a                                  & & \text{by D2}
\end{align*}

{\bf Note.}
The above shows you how to take a fact and produce another that involves
\lq\lq for some ....''.
Basically if you have a propositional formula 
$P(x)$ where $x$ is a variable, 
then if $P(v)$ is true where $v$ is a value, then
\lq\lq $P(x)$ is true for some value for x'' must also be true (duh).
It's pretty obvious right?

It's the same as saying if 
\[
\text{\lq\lq I have a pebble in my pocket''}
\]
then of course 
\[
\text{ \lq\lq I have an $x$ in my pocket for some $x$. }
\]
Right?
This is an \lq\lq axiom'' or rule in logic meaning to say
that this way of deducing a new fact is allowed because it models
the way human beings think.
Because this axiom produces a new fact, it's also
called an {\bf inference rule}.

Note that the \lq\lq opposite'' of that is not true!
Just because I can say that \lq\lq 
I have an $x$ in my pocket
 for some $x$'',
it does not mean that \lq\lq I have a pebble in my pocket''
because what I have in my pocket might very well be my pet lizard.

This is basically what you see in your discrete math class
as an axiom in logic:
\begin{align*}
           &P(a) \\
\THEREFORE &\exists x (P(x))
\end{align*}
This inference rule is called existential generalization.
From now on we'll call it EG.
So you should write the proof like this:
\begin{align*}
           1 a &= a                                     & & \text{by M4} \\
\THEREFORE 1 c &= a \text{ for some integer $c (= a)$ } & & \text{by EG} \\
\THEREFORE 1   &\mid a                                  & & \text{by D2}
\end{align*}

{\bf Note.}
Note that the only reason why proofs at an undergraduate level are
written so tediously is because you have to learn how to think and
argue logically and precisely.
The above format allows you to check the correctness of your logic.
Papers written even in research journals are actually {\it not} 
written in the above format. 
For instance in a paper one would write:
\begin{enumerate}
\item[]
{\it Since 1a = a, we have $1c = a$ for some integer $c$
and hence by definition $1 \mid a$, i.e. $1$ divides $a$.}
\end{enumerate}
or even
\begin{enumerate}
\item[] {\it Since 1a = a, by definition $1$ divides $a$.}
\end{enumerate}

{\bf Note.}
The application of \lq\lq rule'' D1 or D2 is not a deduction
(or inference).
It's just a linguistic translation of notation and definition.


\newpage
Let $M' = (\Sigma, Q', q_0' \delta', F')$
and $M'' = (\Sigma, Q', q_0' \delta', F')$ be two DFAs.
From the$\delta'$ from $M'$,  we have $\delta'^*$.
Likewise we have a $\delta''^*$ from the $\delta''$ of $M''$.
In both the union and intersection construction we define
a $\delta : (Q' \times Q'') \times \Sigma \rightarrow (Q' \times Q'')$
to be
\[
\delta((q',q''), c) = (\delta'(q', c), \delta''(q'',c))
\]
for $q' \in Q'$, $q'' \in Q''$, and $c \in \Sigma$.
Note that $\delta$ depends on $\delta', \delta''$.

This involves the cross product. Here are some basic facts about the
cross product:
\begin{enumerate}
\item[CP1] $(x,y) \in X \times Y$ can be replaced by
           $x \in X, y \in Y$.
\item[CP2] $x \in X, y \in Y$
           can be replaced by
           $(x,y) \in X \times Y$.
\item[CP3] $(x,y) = (x',y')$
           can be replaced by 
           $x=x', y=y'$
\item[CP4] $x=x', y=y'$
           can be replaced by 
           $(x,y) = (x',y')$.
\end{enumerate}
You will be proving something about the intersection construction
so here are some basic things about intersections:
\begin{enumerate}
\item[IN1] $x \in X \cap Y$ 
           can be replaced by
           $x \in X, x \in Y$.
\item[IN2] $x \in X, x \in Y$
           can be replaced by
           $x \in X \cap Y$
\end{enumerate}
[Don't forget that in CS and Math, a comma \lq\lq,'' means \lq\lq and''.]

\newpage
Q2. Assuming the given $M', M''$ above. 
{\bf 
Suppose $\delta'$ is the transition for $M'$
and $\delta''$ is the transition function for $M''$.
We define $\delta$ by 
\[
\delta((q', q''), c)
\]
where $q'\in Q'$, $q'' \in Q''$ and $c \in \Sigma$.
Associated to $\delta', \delta'', \delta''$, 
we have the functions $\delta'^*, \delta''^*, \delta^*$ respectively.
}
Prove that
if $q' \in Q', q'' \in Q'', x \in \Sigma^*$, then
\[
\delta^*((q', q''), x)
=
(
\delta'^*(q', x),
\delta''^*(q', x)
)
\]
[Formulate a $P(n)$ and prove the above by mathematical induction.

{\bf Although not necessary, it might be a good idea to prove the following
Lemma: If $\delta$ is any transition function and $\delta^*$
the associated function, then for $c \in \Sigma$, we have
$\delta^*(q, c) = \delta(q, c)$ for any $q \in Q$.
A Lemma is a small fact that is used in a more important theorem.
It's equivalent to the helper function for programmers.
The proof of this Lemma is about 3 computations long.
The Lemma is used twice in the proof of the result, so if you 
don't have this fact, you would have to more or less go through the 
proof of the Lemma twice in the proof of the actual statement that
we want to prove.
So the Lemma simply allows us to cleanup the main proof.
}]

\SOLUTION
Before we prove the above statement we will prove the following Lemma that we 
use later:

{\bf Lemma.}
Let $\delta: Q \times \Sigma \rightarrow Q$ be any transition function and 
$\delta^*: Q \times \Sigma^* \rightarrow Q$ be the associated function.
Then if $q \in Q$ and $c \in \Sigma$ 
\[
\delta^*(q, c) = \delta(q, c)
\]


{\it Proof of Lemma.} 
Let $c \in \Sigma$ (i.e. $c \in \Sigma^*$ of length 1).
We have the following:
\begin{align*}
\delta^*(q, c)
&= \delta^*(q, c\ep)           & & \text{by C7}\\
&= \delta^*(\delta(q,c), \ep)            & & \text{by DS3} \\
&= \delta(q, c)                & & \text{by DS1}
\end{align*}
The Lemma is proved.
QED.


We will prove the above statement by Mathematical Induction.
For $n \geq 1$, we define
\[
P(n):
\text{
If $x\in \Sigma^*$ with $|x| =n$, then ...
}
\]

\underline{Base case.}
\begin{align*}
|x| = 0                                                     & & \\
x = \ep                                                     & & \text{by L5, $\alpha$}\\
\delta^*((q',q''),x)=\delta^*((q',q''),\ep)& & \text{by $\alpha$} \\
\delta^*((q',q''),\ep)=(q',q'')                             & & \text{by DS1}\\
(q',q'')=(\delta'^*(q',\ep),q)                              & & \text{by DS2}\\
(\delta^*(q',\ep),q)=(\delta'^*(q',\ep),\delta''^*(q'',\ep))& & \text{by DS2}\\
(\delta'^*(q',\ep),\delta''^*)(q'',\ep))=(\delta'^*(q',x),\delta''^*(q'',x))                                                & & \text{by $\alpha$}
\end{align*}


\underline{Inductive case.} 
\begin{align*}
|x| = n + 1                                                                                                     & & \\
|x|\ne 0,x\ne\ep                                                                                                & & \text{by L6} \\
x=x'x''\text{ for some }x'\in\Sigma,\text{ }x''\in\Sigma^*                                                      & & \text{by C8, $\emptyset$} \\
|x'|=1                                                                                                          & & \text{by L2} \\
|x''| = n                                                                                                       & & \text{by L2} \\
\delta^*(q'q'',x)=\delta^*(q'q'',x'x'')                                                                         & & \text{by $\emptyset$} \\
\delta^*(q'q'',x'x'')=(\delta^*(\delta^*(q'q''),x'),x'')                                                        & & \text{by Lemma} \\
(\delta^*(\delta^*(q'q''),x'),x'')=(\delta^*(\delta'(q',x')),\delta''(q'',x')),x'')                             & & \text{by definition of $\delta$} \\
(\delta^*(\delta'(q',x')),\delta''(q'',x')),x'')=(\delta'^*(\delta'(q',x'),x''),\delta''^*(\delta''(q'',x'),x'')) & & \text{by Assumption} \\
(\delta'^*(\delta'(q',x'),x''),\delta''^*(\delta''(q'',x'),x''))=(\delta'^*(q',x'x''),\delta''^*(q'',x'x''))    & & \text{by DS4} \\
(\delta'^*(q',x'x''),\delta''^*(q'',x'x''))=(\delta'^*(q',x),\delta''^*(q'',x))                                 & & \text{by $\emptyset$}
\end{align*}

Hence by Mathematical Induction for all $n \geq 1$
\[
P(n+1)\text{ is true.}
\]
We conclude that if $x \in \Sigma^*$, then
\[
(\delta^*((q',q''),x))=(\delta'^*(q',x),\delta''^*(q'',x))  
\]




\newpage
With the result in Q2 we can now prove that the intersection
construction 
is absolutely true for any pair of DFAs in the following sense:

Q3. 
Let $M' = (\Sigma, Q', q_0', \delta', F')$
and $M'' = (\Sigma, Q'', q_0'', \delta'', F'')$ be two DFAs.
Define a new DFA $M = (\Sigma, Q' \times Q'', (q'_0, q''_0), 
\delta, F' \times F'')$ where
\[
\delta((q',q''), c) = (\delta'(q', c), \delta''(q'',c))
\]
for $q' \in Q'$, $q'' \in Q''$, and $c \in \Sigma$.
Then
\[
L(M) = L(M') \cap L(M'')
\]

[Recall: Two sets $X$ and and $Y$ are the same i.e. $X = Y$,
if $X \subseteq Y$ and $Y \subseteq X$.
This is the formal definition of set equality.
To prove that $X \subseteq Y$, you need to prove
$x \in X \implies x \in Y$.
To prove that $Y \subseteq X$, you need to prove 
$x \in Y \implies x \in X$.
]


\SOLUTION
\begin{align*}
a &\mid b \\
\THEREFORE ax &= b \text{ for some integer $x$}       & & \text{by D1} \\
\THEREFORE (ax)c &= bc \text{ for some integer $x$}   & & \text{by F1} \\
\THEREFORE a(xc) &= bc \text{ for some integer $x$}   & & \text{by M2} \\
\THEREFORE a(xc) &= bc \text{ for some integer $xc$}  & & \text{by M1} \\
\THEREFORE ay &= bc \text{ for some integer $y(=xc)$} & & \text{by EG} \\
\THEREFORE a &\mid bc                                 & & \text{by D2} \\
\end{align*}


\end{document}
