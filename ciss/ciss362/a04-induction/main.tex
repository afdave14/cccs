\input{myassignmentpreamble}
\input{yliow}
\input{ciss362}
\renewcommand\TITLE{Assignment 4}

% CHANGE THE NEXT TWO LINES
\renewcommand\AUTHOR{John Doe ?}
\renewcommand\EMAIL{jdoe1@gmail.com ?}

\begin{document}
\topmatter

We will prove something about languages.

First let me fix (formally) various concepts.
We fix an alphabet $\Sigma$ throughout.
For the following $x,y,z$ are words in $\Sigma^*$.
The following are some basic facts about the word concatenation operator.
\begin{enumerate}
\item[C1] $xy$ is a word in $\Sigma^*$
\item[C2] $(xy)z$ can be rewritten as $x(yz)$
\item[C3] $x(yz)$ can be rewritten as $(xy)z$
\item[C4] $\ep x$ can be rewritten as $x$
\item[C5] $x$ can be rewritten as $\ep x$
\item[C6] $x\ep$ can be rewritten as $x$
\item[C7] $x$ can be rewritten as $x\ep$
\item[C8] If $x \in \Sigma^*$ is not $\ep$, then 
$x = x'x''$ for some $x' \in \Sigma$ and $x'' \in \Sigma^*$.

[This is formalizing the idea that 
\lq\lq a nonempty word always begins with a symbol from the alphabet''.
It's extremely important to be able to speak the formal
mathematical language and yet be able to translate the formal 
syntax into something simple.]
\end{enumerate}

\newpage
First we define formally the length function.
Let $x$ be a word over $\Sigma^*$.
\begin{enumerate}
\item[L1] If $x = \ep$, then we define
\[
|x| = 0
\]
\item[L2] If $x \neq \ep$, then $x = x' \cdot x''$ where
$x \in \Sigma$ (i.e. $x$ is a symbol in our alphabet)
and $x'' \in \Sigma^*$ (i.e. $x''$ is a word over $\Sigma$).
We define 
\[
|x| = 1 + |x''|
\]
\end{enumerate}
As an example, let me compute $|aba|$.
\begin{align*}
|aba| 
&= |a \cdot ba|                           \\
&= 1 + |ba|              & & \text{by L2} \\
&= 1 + |b \cdot a|                        \\
&= 1 + 1 + |a|           & & \text{by L2} \\
&= 1 + 1 + |a \cdot \ep| & &              \\
&= 1 + 1 + 1 + |\ep|     & & \text{by L2} \\
&= 1 + 1 + 1 + 0         & & \text{by L1} \\
&= 3 
\end{align*}

You may assume following properties about the length function:
\begin{enumerate}
\item[L3] $|x| \geq 0$
\item[L4] $|x|$ is an integer 
\end{enumerate}

Note that the \lq\lq other definition'' of length 
\[
|x| = \text{count of the number of symbols in $x$}
\]
although easier to understand (for human beings) is not as precise:
\lq\lq the number of symbols in $x$'' is not strictly mathematical.
To illustrate this very clearly, the definition above
\[
|x| =
\begin{cases}
0       & \text{ if $x = \ep$} \\
1 + |z| & \text{ if $x = yz$ for $y \in \Sigma, z \in \Sigma^*$}
\end{cases}
\]
is recursive (because $|x|$ depends on $|z|$) and hence can be programmed
immediately in C++:

\begin{Verbatim}[frame=single]
#include <iostream>

int len(char * p)
{
    if (p[0] == '\0') // BASE CASE
    {
        return 0;
    }
    else // RECURSIVE CASE
    {
        return 1 + len(p + 1);
    }
}

int main()
{
    char x[] = "abc";
    std::cout << len(x) << std::endl;
    return 0;
}
\end{Verbatim}

This is the reason why recursive thinking is so important.
A recursive fact can be proven to be absolutely true using
Math because of the presence of mathematical induction.
After it's been proven, you can immediately program it using
recursion.

On the other hand, 
\[
|x| = \text{count of the number of symbols in $x$}
\]
is more like performing a scan of the characters manually and therefore looks
executing a loop:
\begin{Verbatim}[frame=single]
#include <iostream>

int len(char * p)
{
    int count = 0;
    while (p[count] != '\0')
    {
        count++;
    }
    return count;
}

int main()
{
    char x[] = "abc";
    std::cout << len(x) << std::endl;
    return 0;
}
\end{Verbatim}

There is no corresponding mathematical concept of loops
and therefore no corresponding mathematical proof technique to prove
correctness of algorithms in a concise manner.

Of course for something as simple of the above, it's no big deal either way.
However for really complex and critical systems, correctness must be
proven or at least verified to some extent.

Read the above two definitions and then read their corresponding code.
Make sure you really understand the difference between the two.
Learn to love recursion.

\newpage
Now for the reverse function. 
The {\bf reverse function} 
$(\cdot)^R$ on a word $x$ is defined as follows:
\begin{enumerate}
\item[R1] If $x = \ep$. We define
\[
\ep^R = \ep
\]
\item[R2] If $x \neq \ep$, then $x = x'\cdot x''$ where
$x' \in \Sigma$ ($x'$ is a symbol in our alphabet $\Sigma$) 
and $x'' \in \Sigma^*$ (i.e., $x''$ is a word over $\Sigma$). 
We define
\[
x^R = (x'')^R \cdot x'
\]
\end{enumerate}
As an example, let me compute $(abb)^R$.
\begin{align*}
(abb)^R
&= (a \cdot bb)^R                                   \\
&= (bb)^R \cdot a                  & & \text{by R2} \\
&= \left( (b\cdot b)^R \right) \cdot a & &              \\
&= \left( b^R \cdot b \right) \cdot a             & & \text{by R2} \\
&= \left( (b\cdot \ep)^R \cdot b \right) \cdot a  & &              \\
&= (\ep)^R \cdot b \cdot b \cdot a & & \text{by R2} \\
&= \ep \cdot b \cdot b \cdot a     & & \text{by R1} \\
&= bba                                              \\
\end{align*}

In terms of C++, using the above recursive definition of the
reverse function we have this (I'm using C++ strings for this example
since cutting and concatenating strings are easier with the C++ string
class):

\begin{Verbatim}[frame=single]
#include <iostream>
#include <string>

std::string reverse(const std::string & s)
{
    if (s == "") // BASE CASE
    {
        return "";
    }
    else // RECURSIVE CASE
    {
        char y = s[0];
        std::string z = s.substr(1);
        return reverse(z) + y;
    }
}

int main()
{
    std::string x("abc");
    std::cout << reverse(x) << std::endl;    
    return 0;
}
\end{Verbatim}

The other definition of reverse, i.e. \lq\lq the reverse of 
$x$ is the word which is the same as $x$ but with the
symbols in reverse order'' is not precise mathematically.
The code would look something like this:
\begin{Verbatim}[frame=single]
#include <iostream>
#include <string>

std::string reverse(const std::string & s)
{
    std::string t = "";
    for (size_t i = 0; i < s.length(); i++)
    {
        t.push_back(s[s.length() - i - 1]);
    }
    return t;
}

int main()
{
    std::string x("abc");
    std::cout << reverse(x) << std::endl;
    
    return 0;
}
\end{Verbatim}

\newpage

Let $\Sigma$ be an alphabet.
Let $x$ and $y$ be words in $\Sigma^*$.
We want to prove that 
\[
P:
\text{
If $x,y$ are words in $\Sigma^*$, then
$(xy)^R = y^R x^R$
}
\]
Instead of proving $P$ directly, 
we will prove this by mathematical induction,
inducting on the length of $x$.
Therefore we let $P(n)$ be this statement:
\[
P(n): 
\text{
If $x,y$ are words in $\Sigma^*$ with $|x| = n$, then
$(xy)^R = y^R x^R$
}
\]
Since the length of $x$ can be any integer $n \geq 0$,
our base case is when $n = 0$.


\newpage



Q1. We know that if $x = \ep$, then $|x| = 0$.
Now for the converse:

Prove the following:
Let $x$ be a word in $\Sigma^*$.
If $|x| = 0$, then $x = \ep$.

\SOLUTION



\newpage

Q2. Let $x$ be a word in $\Sigma^*$.
Then 
\[
|x| \neq 0 \iff x \neq \ep
\]

\SOLUTION
We already know that $|x| = 0 \iff x = \ep$ (see Q1).
Hence
\[
|x| \neq 0 \iff x \neq \ep
\]

(You don't have to prove anything here. I've done everything.)


\newpage
Let's get back to the main problem.
Recall that we are trying to prove
\[
P(n): 
\text{
If $x,y$ are words in $\Sigma^*$ with $|x| = n$, then
$(xy)^R = y^R x^R$
}
\]
Let's us go for the base case.

Q3. 
Prove that $P(0)$ is true.

\SOLUTION


\newpage

Recall that we are trying to prove
\[
P(n): 
\text{
If $x,y$ are words in $\Sigma^*$ with $|x| = n$, then
$(xy)^R = y^R x^R$
}
\]
We are done with the base case.
The only thing left is the inductive case.
Note that there are two forms of mathematical induction.
To prove that $P(n)$ is true for all $n \geq 0$.
You can prove the following two statesments hold:
\begin{enumerate}
\item $P(0)$ is true
\item If $P(n)$ is true, then $P(n+1)$ is true.
\end{enumerate}
Or you can prove the following holds:
\begin{enumerate}
\item $P(0)$ is true
\item If $P(0), P(1), ..., P(n)$ are true, then $P(n+1)$ is true.
\end{enumerate}
The second form seems weaker since you need to assume more, 
i.e. you need to assume that $P(0), P(1), ..., P(n)$ are all true.
That's why the second form of mathematical induction is called
the weak form of induction.
The first form is called the strong form.
It turns out that they are equally powerful.
We will be using the weak form for the proof
of our inductive case.
\newpage

Q4. 
Recall that 
\[
P(n): 
\text{
If $x,y$ are words in $\Sigma^*$ with $|x| = n$, then
$(xy)^R = y^R x^R$
}
\]

Let $n \geq 0$.
Assume that $P(k)$ is true for $k = 0, 1, \ldots, n$.
Prove that $P(n+1)$ is true.

\SOLUTION


\newpage

Altogether for the following statement:
\[
P(n): 
\text{
If $x,y$ are words in $\Sigma^*$ with $|x| = n$, then
$(xy)^R = y^R x^R$
}
\]
we have shown that
\begin{enumerate}
\item $P(0)$ is true
\item If $P(0), P(1), \ldots, P(n)$ are true, then $P(n+1)$ is true.
\end{enumerate}
By mathematical induction, $P(n)$ must be true for all $n \geq 0$.
Therefore the following must be true:

If $x,y$ are words in $\Sigma^*$, then
\[
(xy)^R = y^R x^R
\]
QED.

\end{document}
