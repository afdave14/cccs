\begin{align*}
           1 a &= a                                     & & \text{by M4} \\
\THEREFORE 1 c &= a \text{ for some integer $c (= a)$ } & & \\
\THEREFORE 1   &\mid a                                  & & \text{by D2}
\end{align*}

{\bf Note.}
The above shows you how to take a fact and produce another that involves
\lq\lq for some ....''.
Basically if you have a propositional formula 
$P(x)$ where $x$ is a variable, 
then if $P(v)$ is true where $v$ is a value, then
\lq\lq $P(x)$ is true for some value for x'' must also be true (duh).
It's pretty obvious right?

It's the same as saying if 
\[
\text{\lq\lq I have a pebble in my pocket''}
\]
then of course 
\[
\text{ \lq\lq I have an $x$ in my pocket for some $x$. }
\]
Right?
This is an \lq\lq axiom'' or rule in logic meaning to say
that this way of deducing a new fact is allowed because it models
the way human beings think.
Because this axiom produces a new fact, it's also
called an {\bf inference rule}.

Note that the \lq\lq opposite'' of that is not true!
Just because I can say that \lq\lq 
I have an $x$ in my pocket
 for some $x$'',
it does not mean that \lq\lq I have a pebble in my pocket''
because what I have in my pocket might very well be my pet lizard.

This is basically what you see in your discrete math class
as an axiom in logic:
\begin{align*}
           &P(a) \\
\THEREFORE &\exists x (P(x))
\end{align*}
This inference rule is called existential generalization.
From now on we'll call it EG.
So you should write the proof like this:
\begin{align*}
           1 a &= a                                     & & \text{by M4} \\
\THEREFORE 1 c &= a \text{ for some integer $c (= a)$ } & & \text{by EG} \\
\THEREFORE 1   &\mid a                                  & & \text{by D2}
\end{align*}

{\bf Note.}
Note that the only reason why proofs at an undergraduate level are
written so tediously is because you have to learn how to think and
argue logically and precisely.
The above format allows you to check the correctness of your logic.
Papers written even in research journals are actually {\it not} 
written in the above format. 
For instance in a paper one would write:
\begin{enumerate}
\item[]
{\it Since 1a = a, we have $1c = a$ for some integer $c$
and hence by definition $1 \mid a$, i.e. $1$ divides $a$.}
\end{enumerate}
or even
\begin{enumerate}
\item[] {\it Since 1a = a, by definition $1$ divides $a$.}
\end{enumerate}

{\bf Note.}
The application of \lq\lq rule'' D1 or D2 is not a deduction
(or inference).
It's just a linguistic translation of notation and definition.
