\input{myassignmentpreamble}
\input{yliow}
\input{ciss362}
\renewcommand\TITLE{Assignment 3}

\begin{document}
\topmatter

The following are some properties of integers.
In the following $x, y, z, w$ are integers.
First we have the basic rules for addition:
\begin{enumerate}
\item [A1.] $x + y$ is an integer.
\item [A2.] $(x + y) + z$ can be rewritten as $x + (y + z)$.
\item [A3.] $x + (y + z)$ can be rewritten as $(x + y) + z$.
\item [A4.] $x + (-x)$ can be rewritten as $0$.
\item [A5.] $0$ can be rewritten as $x + (-x)$.
\item [A6.] $0 + x$ can be rewritten as $x$.
\item [A7.] $x$ can be rewritten as $0 + x$.
\item [A8.] $x + 0$ can be rewritten as $x$.
\item [A9.] $x$ can be rewritten as $x + 0$.
\item [A10.] $x + y$ can be rewritten as $y + x$.
\end{enumerate}
(A = addition.)
Now for the multiplication rules:
\begin{enumerate}
\item [M1.] $xy$ is an integer.
\item [M2.] $(xy)z$ can be rewritten as $x(yz)$.
\item [M3.] $x(yz)$ can be rewritten as $(xy)z$.
\item [M4.] $1x$ can be rewritten as $x$.
\item [M5.] $x$ can be rewritten as $1x$.
\item [M6.] $x1$ can be rewritten as $x$.
\item [M7.] $x$ can be rewritten as $x1$.
\item [M8.] $xy$ can be rewritten as $yx$.
\end{enumerate}
(M = multiplication.)
Here are the rules involving both addition and multiplication
\begin{enumerate}
\item [AM1.] $x(y + z)$ can be rewritten as $xy + xz$
\item [AM2.] $xy + xz$ can be rewritten as $x(y + z)$
\end{enumerate}
(AM = addition and multiplication,)
Here are some facts that you may assume.
There are facts that can be derived from the above rules
\begin{enumerate}
\item [F1.] If $x = y$, then $xz = yz$.
\item [F3.] If $x = y$, then $x + z = y + z$.
\item [F3.] $0x$ can be replaced by $0$.
\item [F4.] $x0$ can be replaced by $0$.
\item [F5.] If $x = y$ and $z = w$, then $x + z = y + w$.
\end{enumerate}
(F = facts.)
We want to prove some basic facts divisibility.
Given two integer $a, b$ we say that $a$ divides $b$, and we write $a \mid b$
is there is an integer $c$ such that $ac = b$.
This of course means that

\begin{enumerate}
\item [D1] \lq\lq $a \mid b$'' 
           can be replaced by \lq\lq $ac = b$ for some integer $c$''.
\item [D2] \lq\lq $ac = b$ for some integer $c$'' 
           can be replaced by \lq\lq $a \mid b$''.
\end{enumerate}
(D = definition.)

Several proofs are already provided.
make sure you study them to see how I want you to write the proofs.
I'v simplified the proofs to make it easier to understand so that
we're not overly caught up with formal logic.

\newpage

Q1. Let $a$ be an integer. Prove that $1 \mid a$.

\SOLUTION



\newpage


Q2. Let $a$ be an integer. Prove that $a \mid a$.

\SOLUTION


\newpage


Q3. Let $a,b,c$ be integers.
Prove that if $a \mid b$, then $a \mid (bc)$.

\SOLUTION


\newpage


Q4. Let $a,b,c$ be integers.
Prove that if $a \mid b$ and $b \mid c$, then $a \mid c$.

\SOLUTION


\newpage


Q5. Let $a,b,c$ be integers.
Prove that if $a \mid b$ and $a \mid c$, then $a \mid (b + c)$.

\SOLUTION


\newpage


\end{document}
