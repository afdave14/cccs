\input{myassignmentpreamble}
\input{yliow}
\input{ciss362}
\renewcommand\TITLE{Assignment 3}

\begin{document}
\topmatter

The following are some properties of integers.
In the following $x, y, z, w$ are integers.
First we have the basic rules for addition:
\begin{enumerate}
\item [A1.] $x + y$ is an integer.
\item [A2.] $(x + y) + z$ can be rewritten as $x + (y + z)$.
\item [A3.] $x + (y + z)$ can be rewritten as $(x + y) + z$.
\item [A4.] $x + (-x)$ can be rewritten as $0$.
\item [A5.] $0$ can be rewritten as $x + (-x)$.
\item [A6.] $0 + x$ can be rewritten as $x$.
\item [A7.] $x$ can be rewritten as $0 + x$.
\item [A8.] $x + 0$ can be rewritten as $x$.
\item [A9.] $x$ can be rewritten as $x + 0$.
\item [A10.] $x + y$ can be rewritten as $y + x$.
\end{enumerate}
(A = addition.)
Now for the multiplication rules:
\begin{enumerate}
\item [M1.] $xy$ is an integer.
\item [M2.] $(xy)z$ can be rewritten as $x(yz)$.
\item [M3.] $x(yz)$ can be rewritten as $(xy)z$.
\item [M4.] $1x$ can be rewritten as $x$.
\item [M5.] $x$ can be rewritten as $1x$.
\item [M6.] $x1$ can be rewritten as $x$.
\item [M7.] $x$ can be rewritten as $x1$.
\item [M8.] $xy$ can be rewritten as $yx$.
\end{enumerate}
(M = multiplication.)
Here are the rules involving both addition and multiplication
\begin{enumerate}
\item [AM1.] $x(y + z)$ can be rewritten as $xy + xz$
\item [AM2.] $xy + xz$ can be rewritten as $x(y + z)$
\end{enumerate}
(AM = addition and multiplication,)
Here are some facts that you may assume.
There are facts that can be derived from the above rules
\begin{enumerate}
\item [F1.] If $x = y$, then $xz = yz$.
\item [F3.] If $x = y$, then $x + z = y + z$.
\item [F3.] $0x$ can be replaced by $0$.
\item [F4.] $x0$ can be replaced by $0$.
\item [F5.] If $x = y$ and $z = w$, then $x + z = y + w$.
\end{enumerate}
(F = facts.)
We want to prove some basic facts divisibility.
Given two integer $a, b$ we say that $a$ divides $b$, and we write $a \mid b$
is there is an integer $c$ such that $ac = b$.
This of course means that

\begin{enumerate}
\item [D1] \lq\lq $a \mid b$'' 
           can be replaced by \lq\lq $ac = b$ for some integer $c$''.
\item [D2] \lq\lq $ac = b$ for some integer $c$'' 
           can be replaced by \lq\lq $a \mid b$''.
\end{enumerate}
(D = definition.)

Several proofs are already provided.
make sure you study them to see how I want you to write the proofs.
I'v simplified the proofs to make it easier to understand so that
we're not overly caught up with formal logic.

\newpage

Q1. Let $a$ be an integer. Prove that $1 \mid a$.

\SOLUTION
\begin{align*}
           1 a &= a                                     & & \text{by M4} \\
\THEREFORE 1 c &= a \text{ for some integer $c (= a)$ } & & \\
\THEREFORE 1   &\mid a                                  & & \text{by D2}
\end{align*}

{\bf Note.}
The above shows you how to take a fact and produce another that involves
\lq\lq for some ....''.
Basically if you have a propositional formula 
$P(x)$ where $x$ is a variable, 
then if $P(v)$ is true where $v$ is a value, then
\lq\lq $P(x)$ is true for some value for x'' must also be true (duh).
It's pretty obvious right?

It's the same as saying if 
\[
\text{\lq\lq I have a pebble in my pocket''}
\]
then of course 
\[
\text{ \lq\lq I have an $x$ in my pocket for some $x$. }
\]
Right?
This is an \lq\lq axiom'' or rule in logic meaning to say
that this way of deducing a new fact is allowed because it models
the way human beings think.
Because this axiom produces a new fact, it's also
called an {\bf inference rule}.

Note that the \lq\lq opposite'' of that is not true!
Just because I can say that \lq\lq 
I have an $x$ in my pocket
 for some $x$'',
it does not mean that \lq\lq I have a pebble in my pocket''
because what I have in my pocket might very well be my pet lizard.

This is basically what you see in your discrete math class
as an axiom in logic:
\begin{align*}
           &P(a) \\
\THEREFORE &\exists x (P(x))
\end{align*}
This inference rule is called existential generalization.
From now on we'll call it EG.
So you should write the proof like this:
\begin{align*}
           1 a &= a                                     & & \text{by M4} \\
\THEREFORE 1 c &= a \text{ for some integer $c (= a)$ } & & \text{by EG} \\
\THEREFORE 1   &\mid a                                  & & \text{by D2}
\end{align*}

{\bf Note.}
Note that the only reason why proofs at an undergraduate level are
written so tediously is because you have to learn how to think and
argue logically and precisely.
The above format allows you to check the correctness of your logic.
Papers written even in research journals are actually {\it not} 
written in the above format. 
For instance in a paper one would write:
\begin{enumerate}
\item[]
{\it Since 1a = a, we have $1c = a$ for some integer $c$
and hence by definition $1 \mid a$, i.e. $1$ divides $a$.}
\end{enumerate}
or even
\begin{enumerate}
\item[] {\it Since 1a = a, by definition $1$ divides $a$.}
\end{enumerate}

{\bf Note.}
The application of \lq\lq rule'' D1 or D2 is not a deduction
(or inference).
It's just a linguistic translation of notation and definition.

\newpage


Q2. Let $a$ be an integer. Prove that $a \mid a$.

\SOLUTION
Before we prove the above statement we will prove the following Lemma that we 
use later:

{\bf Lemma.}
Let $\delta: Q \times \Sigma \rightarrow Q$ be any transition function and 
$\delta^*: Q \times \Sigma^* \rightarrow Q$ be the associated function.
Then if $q \in Q$ and $c \in \Sigma$ 
\[
\delta^*(q, c) = \delta(q, c)
\]


{\it Proof of Lemma.} 
Let $c \in \Sigma$ (i.e. $c \in \Sigma^*$ of length 1).
We have the following:
\begin{align*}
\delta^*(q, c)
&= \delta^*(q, c\ep)           & & \text{by C7}\\
&= \delta^*(\delta(q,c), \ep)            & & \text{by DS3} \\
&= \delta(q, c)                & & \text{by DS1}
\end{align*}
The Lemma is proved.
QED.


We will prove the above statement by Mathematical Induction.
For $n \geq 1$, we define
\[
P(n):
\text{
If $x\in \Sigma^*$ with $|x| =n$, then ...
}
\]

\underline{Base case.}
\begin{align*}
|x| = 0                                                     & & \\
x = \ep                                                     & & \text{by L5, $\alpha$}\\
\delta^*((q',q''),x)=\delta^*((q',q''),\ep)& & \text{by $\alpha$} \\
\delta^*((q',q''),\ep)=(q',q'')                             & & \text{by DS1}\\
(q',q'')=(\delta'^*(q',\ep),q)                              & & \text{by DS2}\\
(\delta^*(q',\ep),q)=(\delta'^*(q',\ep),\delta''^*(q'',\ep))& & \text{by DS2}\\
(\delta'^*(q',\ep),\delta''^*)(q'',\ep))=(\delta'^*(q',x),\delta''^*(q'',x))                                                & & \text{by $\alpha$}
\end{align*}


\underline{Inductive case.} 
\begin{align*}
|x| = n + 1                                                                                                     & & \\
|x|\ne 0,x\ne\ep                                                                                                & & \text{by L6} \\
x=x'x''\text{ for some }x'\in\Sigma,\text{ }x''\in\Sigma^*                                                      & & \text{by C8, $\emptyset$} \\
|x'|=1                                                                                                          & & \text{by L2} \\
|x''| = n                                                                                                       & & \text{by L2} \\
\delta^*(q'q'',x)=\delta^*(q'q'',x'x'')                                                                         & & \text{by $\emptyset$} \\
\delta^*(q'q'',x'x'')=(\delta^*(\delta^*(q'q''),x'),x'')                                                        & & \text{by Lemma} \\
(\delta^*(\delta^*(q'q''),x'),x'')=(\delta^*(\delta'(q',x')),\delta''(q'',x')),x'')                             & & \text{by definition of $\delta$} \\
(\delta^*(\delta'(q',x')),\delta''(q'',x')),x'')=(\delta'^*(\delta'(q',x'),x''),\delta''^*(\delta''(q'',x'),x'')) & & \text{by Assumption} \\
(\delta'^*(\delta'(q',x'),x''),\delta''^*(\delta''(q'',x'),x''))=(\delta'^*(q',x'x''),\delta''^*(q'',x'x''))    & & \text{by DS4} \\
(\delta'^*(q',x'x''),\delta''^*(q'',x'x''))=(\delta'^*(q',x),\delta''^*(q'',x))                                 & & \text{by $\emptyset$}
\end{align*}

Hence by Mathematical Induction for all $n \geq 1$
\[
P(n+1)\text{ is true.}
\]
We conclude that if $x \in \Sigma^*$, then
\[
(\delta^*((q',q''),x))=(\delta'^*(q',x),\delta''^*(q'',x))  
\]

\newpage


Q3. Let $a,b,c$ be integers.
Prove that if $a \mid b$, then $a \mid (bc)$.

\SOLUTION
\begin{align*}
a &\mid b \\
\THEREFORE ax &= b \text{ for some integer $x$}       & & \text{by D1} \\
\THEREFORE (ax)c &= bc \text{ for some integer $x$}   & & \text{by F1} \\
\THEREFORE a(xc) &= bc \text{ for some integer $x$}   & & \text{by M2} \\
\THEREFORE a(xc) &= bc \text{ for some integer $xc$}  & & \text{by M1} \\
\THEREFORE ay &= bc \text{ for some integer $y(=xc)$} & & \text{by EG} \\
\THEREFORE a &\mid bc                                 & & \text{by D2} \\
\end{align*}

\newpage


Q4. Let $a,b,c$ be integers.
Prove that if $a \mid b$ and $b \mid c$, then $a \mid c$.

\SOLUTION
Let $x,y$ be words in $\Sigma^*$ with $|x| = n + 1$.

We have the following:
\begin{align*}
           |x| &= n + 1  & & \tag{A} \\ 
\THEREFORE |x| &\geq 1 \\
\THEREFORE |x| &\neq 0 \\
\THEREFORE x &\neq ? & & \text{by Q2} \\
\THEREFORE x &= x'x'' \text{ for some } x' \in \Sigma, x'' \in \Sigma^* 
           & & \text{by ?} 
           \tag{B} \\
\THEREFORE |x| &= 1 + |x''| \text{ for some } x'' \in \Sigma^*
           & & \text{by ?} \\
\THEREFORE n + 1 &= 1 + |x''| \text{ for some } x'' \in \Sigma^*
           & & \text{by A} \\
\THEREFORE |x''| &= n \text{ for some } x'' \in \Sigma^*
\end{align*}

Therefore for some $x' \in \Sigma$ and $x'' \in \Sigma^*$
we have the following:
\begin{align*}
(xy)^R
&= (?x'' y)^R                                   & & \text{by (B)} \\
&= (? \cdot x''y)^R                                               \\
&= (x''y)^R \cdot (?)^R                         & & \text{by $P(1)$}\\
&= \left( y^R \cdot (x'')^R \right) \cdot (?)^R & & \text{by $P(n)$} \\
&= y^R \cdot \left( (x'')^R \cdot (?)^R \right) & & \text{by ?}\\
&= y^R \cdot \left( ?x'' \right)^R              & & \text{by ?}\\
&= ?^R \cdot ?^R                                & & \text{by (B)} 
\end{align*}
Hence $P(n+1)$ holds.
QED.


{\bf Note.}
The intuition behind the proof is that 
given $x$ of length $n + 1$,
we cut it up into $x'$ and $x''$ of lengths 1 and $n$.
We then using the inductive hypothesis $P(1)$ and $P(n)$.
Read over the proof again and make sure the see the strategy
in the proof.
Writing proofs formally is one thing.
Understanding the strategy in a proof is another.
Only by studying lots of proofs and understanding their strategy
then will you really understand how to construct convincing proofs
of your own.

\newpage


Q5. Let $a,b,c$ be integers.
Prove that if $a \mid b$ and $a \mid c$, then $a \mid (b + c)$.

\SOLUTION
From $a \mid b$ we have:
\begin{align*}
a &\mid b  \\
\THEREFORE ax &= b \text{ for some integer $x$} & & \text{by ?} \tag{A}
\end{align*}
From $a \mid c$ we have:
\begin{align*}
a &\mid c  \\
\THEREFORE ay &= ? \text{ for some integer $?$} & & \text{by ?} \tag{B}
\end{align*}
From (A) and (B) we have:
\begin{align*}
           ax + ay &= b + ? \text{ for some integers $x, y$}               \\
\THEREFORE a?  &= b + ? \text{ for some integers $x, y$}     & & \text{by ?} \\
\THEREFORE a?  &= b + ? \text{ for some integer $x + y$}     & & \text{by ?} \\
\THEREFORE az  &= b + ? \text{ for some integers $z (= ?)$}  & & \text{by ?} \\
\THEREFORE ?   &\mid ?                                       & & \text{by ?} 
\end{align*}


\newpage


\end{document}
