\input{myassignmentpreamble}
\input{yliow}
\input{ciss362}

\renewcommand\TITLE{Final Exam}
\renewcommand\AUTHOR{John Doe}
\renewcommand\EMAIL{jdoe@jdoe.com}

\begin{document}
\topmatter

The takehome final exam 
is due Thursday 5PM.
You must turn in a printed pdf in person, not by email.

There are two version of the final exam.
You either attempt Q1-7 or you attempt Q8 but not both.
For Q8, you only need to complete one of the three parts.

You cannot attempt both Q1-Q7 and Q8.
This means that if you have \textit{any} attempt at all for Q1-Q7 
and \textit{any} attempt for Q8, you will get an immediate zero. 

You must writing clearly, concisely, and in complete sentences, using
proper mathematical notation.
In case of ambiguity, I get to choose the meaning of what you're trying to 
say!




\newpage
Q1. Let $\Sigma = \{0,1\}$. Define the following language:
\[
L = 
\{
x \mid
x \text{ contains an equal number of occurrences of $01$ and $10$}
\}
\] 
Either prove $L$ is regular (by constructing a DFA/NFA or a regex) or
prove that it is not regular using the Pumping Lemma for regular languages.

\SOLUTION








\newpage
\newcommand\shuffle{\operatorname{Shuffle}}

Q2. Given languages $L, L'$ (over $\Sigma$), 
define the shuffle of $L,L'$ to be
\[
\shuffle(L,L') = \{x_1 y_1 x_2 y_2 \cdots x_n y_n \mid 
x_i \in L \text{ and } 
y_i \in L' 
\text{ for } i=1,\ldots,n 
\}
\]
Suppose that 
$M = (\Sigma, Q, q_0, \delta, F)$ is a DFA for $L$ and
$M' = (\Sigma, Q', q_0', \delta', F')$ is a DFA for $L'$.
Construct an NFA that accepts $\shuffle(L)$.

[This proves that $\shuffle$ is a closed regular operator, i.e.,
this proves that if $L,L'$ are regular, then 
$\shuffle(L,L')$ is also regular.]


\SOLUTION







\newpage
Q3. Let $\Sigma = \{a,b,c\}$.
Prove that the following language (defined over $\Sigma$)
\[
L = 
\{
uvuvu \mid
u \in \{a,b\}^*, v \in \{b,c\}^*
\}
\]
is not regular.

[HINT: Clean up and then use the pumping lemma for regular languages.]

\SOLUTION







\newpage
Q4. Prove that the following languages are context-free by constructing
the a context-free grammar for $L_1$ and a push-down automata for $L_2$.
\begin{enumerate}[topsep=0in,parsep=0in]
\item[(a)] $L_1 = \{a^m b^m c^n \mid m \geq 0, n \geq 0\}$
\item[(b)] $L_2 = \{a^m b^n c^n \mid m \geq 0, n \geq 0\}$
\end{enumerate}
Finally:
\begin{enumerate}[topsep=0in,parsep=0in]
\item[(c)] What is $L_1 \cap L_2$? [Use set notation.]
\item[(d)] TRUE or FALSE: The intersection of any two context-free languages
gives a context-free language.
\end{enumerate}


\SOLUTION







\newpage
Q5. Prove that 
\[
L = 
\{
a^m b^n 
\mid
m \neq n \text{ and }
2m \neq n
\}
\]
is a context-free language.

\SOLUTION







\newpage
Q6. Suppose $L$ is a context-free language and 
$L'$ is regular.
Show that $L \cap L'$ is a context-free language.
Specifically, if you're given a PDA
diagram of $L$ and a DFA diagram for $L'$,
describe informally how to draw a PDA diagram for $L \cap L'$.

[... it's even better if you can describe the construction of
$L \cap L'$ formally but you don't have to.]

\SOLUTION

\input {q6.tex}




\newpage
\newcommand\ACCEPT{\operatorname{\text{\textsc{Accept}}}}
\newcommand\DFA{{\operatorname{\text{\textsc{DFA1}}}}}
\newcommand\NONEMPTY{\operatorname{\text{\textsc{NonEmpty}}}}
\newcommand\TM{{\operatorname{TM}}}

Q7. (a) A Turing machine $M$ is considered a DFA1 if
the read/write head always moves to the right and it halts
(i.e., enters its $q_\accept$ or $q_\reject$) when 
it reads a space (or blank) character on the input tape.
Note that a DFA1 is like a DFA except that it must have exactly 1
accept state.
(A DFA can have any number of accept states.)
Is the following language:
\[
\ACCEPT_\DFA 
= 
\{ \langle M \rangle \# \langle w \rangle 
\mid
M \text{ is a DFA1 and accepts } w \}
\]
a Turing--decidable language? 
If it is, explain clearly how to construct a Turing decider
(i.e., a Turing machine that always halts) that accepts
the language. Otherwise prove that it's not decidable.

(b) Consider the following language:
\[
\NONEMPTY_\TM = \{
\langle M \rangle 
\#
\langle w \rangle
\mid
M \text{ is a Turing machine and }
L(M) \neq \emptyset
\}
\]
Is $\NONEMPTY_\TM$ a Turing--decidable language?

[This is the only TM question but it actually requires very little
TM knowledge.
You have everything you need to solve this
problem if have a general understanding of TMs and
you have paid attention to the Wednesday and Friday
classes during the last week.]

\SOLUTION


(a)



(b)





\newpage
Q8. This is the question for the second version of the final exam.
You only need to attempt one of the three parts.
Do \textit{not} turn in any work for Q1-Q7 if you plan to turn in 
work for Q8, or you will get an immediate 0.

\begin{enumerate}

\item
Given two words $x, y \in \Sigma^*$, a DFA
$M$ separates $x, y$ if $L$ accepts either $x$ or $y$ but not both.
What is the smallest number of states you need to construct a
DFA
that can separate any pair of distinct words of length $\leq n$?

\item Consider the following problem:
Given a PDA $M$ and an integer $n$, is it true that
$\Sigma^n \subseteq L(M)$?
The corresponding language is
\[
L = \{\langle M \rangle \# \langle n \rangle \mid
M \text{ is a PDA and } \Sigma^n \subseteq L(M)
\}
\]
Is $L$ regular, context-free, Turing--decidable, Turing--recognizable,
not Turing--recognizable?

\item
Either prove 
$\operatorname{P} = \operatorname{NP}$ or
$\operatorname{P} \neq \operatorname{NP}$.

\end{enumerate}

\SOLUTION

\input {q8.tex}


\end{document}
