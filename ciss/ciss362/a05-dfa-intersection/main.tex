\input{myassignmentpreamble}
\input{yliow}
\input{ciss362}
\renewcommand\TITLE{Assignment 5}
\renewcommand\AUTHOR{John Doe}
\renewcommand\EMAIL{jdoe1@jdoe1.com}

\begin{document}
\topmatter

We'll take a break from proofs and do some
DFA construction.

The questions are taken from our textbook.
When I say \lq\lq Textbook Q 1.1'' 
I mean exercise/problem 1.1 in our textbook.

Q1 and Q2 are not graded since answers are provided in the textbook.
Q3 is not graded because it's too easy -- I'm giving you the solution.

For Q4, (a), (b), (d), and (g) will not be graded.
I'm giving you the solution for 4(a) and (g).
The textbook provides solutions for Q4(b) and Q4(d).
Make sure you study my solutions and the author's solutions carefully.

Also, as I mentioned at the beginning of this course, I might not grade
all questions.
However you are strongly encouraged to solve every 
single problems.




\newpage
Q1. Textbook Q 1.1. (Not graded.)

\SOLUTION







\newpage
Q2. Textbook Q 1.2. (Not graded.)

\SOLUTION





\newpage
Q3. Textbook Q 1.3

\SOLUTION





\newpage
Q4. Textbook Q1.4. ((b), (d), and (g) not graded.)

Each language in this question can be written as an intersection
of two languages that can accepted by DFAs.
Given two DFAs, say $M_1$ and $M_2$, 
then it's always possible to design another DFA $M$ such that
\[
L(M) = L(M_1) \cap L(M_2).
\]
See the textbook.

Try to layout the states in the \lq\lq obvious'' way.
(Look at the solution for (a) and (g).)

\SOLUTION


4(a)
Let 
\[
L = \{w \mid \text{$w$ has at least three $a$'s and at least two $b$'s}\}
\]
Note that if 
\begin{align*}
L' &= \{w \mid \text{$w$ has at least three $a$'s} \} \\
L'' &= \{w \mid \text{$w$ has at least two $b$'s} \}
\end{align*}
then
\[
L = L' \cap L''
\]
Therefore we construct DFAs for $L'$ and $L''$ first.

The following $M'$ is a DFA such that $L(M') = L'$:
\begin{center}
\begin{tikzpicture}[shorten >=1pt,node distance=2cm,auto,initial text=]
\node[state,initial]   (q_0)          {$q'_0$};
\node[state]           (q_1) at (2,0) {$q'_1$};
\node[state]           (q_2) at (4,0) {$q'_2$};
\node[state,accepting] (q_3) at (6,0) {$q'_3$};

\path[->] (q_0) edge node {$a$} (q_1)
          (q_0) edge [loop above] node {$b$} ()
          (q_1) edge node {$a$} (q_2)
          (q_1) edge [loop above] node {$b$} ()
          (q_2) edge node {$a$} (q_3)
          (q_2) edge [loop above] node {$b$} ()
          (q_3) edge [loop above] node {$a, b$} ()
;
\end{tikzpicture}
\end{center}
The following $M''$ is a DFA such that $L(M'') = L''$:
\begin{center}
\begin{tikzpicture}[shorten >=1pt,node distance=2cm,auto,initial text=]
\node[state,initial]   (q_0)          {$q''_0$};
\node[state]           (q_1) at (2,0) {$q''_1$};
\node[state,accepting] (q_2) at (4,0) {$q''_2$};

\path[->] (q_0) edge node {$b$} (q_1)
          (q_0) edge [loop above] node {$a$} ()
          (q_1) edge node {$b$} (q_2)
          (q_1) edge [loop above] node {$a$} ()
          (q_2) edge [loop above] node {$a, b$} ()
;
\end{tikzpicture}
\end{center}
The following is a DFA $M$ such that $L(M) = L(M') \cap L(M'')$:
\begin{center}
\begin{tikzpicture}[shorten >=1pt,node distance=2cm,auto,initial text=]
\node[state,initial]   (q_00)          {$(q'_0, q''_0)$};
\node[state]           (q_10) at (3,0) {$(q'_1, q''_0)$};
\node[state]           (q_20) at (6,0) {$(q'_2, q''_0)$};
\node[state]           (q_30) at (9,0) {$(q'_3, q''_0)$};
\node[state]           (q_01) at (0,-4) {$(q'_0, q''_1)$};
\node[state]           (q_11) at (3,-4) {$(q'_1, q''_1)$};
\node[state]           (q_21) at (6,-4) {$(q'_2, q''_1)$};
\node[state]           (q_31) at (9,-4) {$(q'_3, q''_1)$};
\node[state]           (q_02) at (0,-8) {$(q'_0, q''_2)$};
\node[state]           (q_12) at (3,-8) {$(q'_1, q''_2)$};
\node[state]           (q_22) at (6,-8) {$(q'_2, q''_2)$};
\node[state,accepting] (q_32) at (9,-8) {$(q'_3, q''_2)$};

\path[->] (q_00) edge node {$a$} (q_10)
          (q_00) edge node {$b$} (q_01)
          (q_10) edge node {$a$} (q_20)
          (q_10) edge node {$b$} (q_11)
          (q_20) edge node {$a$} (q_30)
          (q_20) edge node {$b$} (q_21)
          (q_30) edge [loop right] node {$a$} ()
          (q_30) edge node {$b$} (q_31)

          (q_01) edge node {$a$} (q_11)
          (q_01) edge node {$b$} (q_02)
          (q_11) edge node {$a$} (q_21)
          (q_11) edge node {$b$} (q_12)
          (q_21) edge node {$a$} (q_31)
          (q_21) edge node {$b$} (q_22)
          (q_31) edge [loop right] node {$a$} ()
          (q_31) edge node {$b$} (q_32)

          (q_02) edge node {$a$} (q_12)
          (q_02) edge [loop below] node {$b$} ()
          (q_12) edge node {$a$} (q_22)
          (q_12) edge [loop below] node {$b$} ()
          (q_22) edge node {$a$} (q_32)
          (q_22) edge [loop below] node {$b$} ()
          (q_32) edge [loop right] node {$a, b$} ()

;
\end{tikzpicture}
\end{center}


\newpage

4(b)


\newpage

4(c)


\newpage

4(d)


\newpage

4(e)


\newpage

4(f)


\newpage

4(g)
Let 
\[
L = \{w \mid \text{$w$ has even length and odd number of $a$'s}\}
\]
Note that if 
\begin{align*}
L' &= \{w \mid \text{$w$ has even length} \} \\
L'' &= \{w \mid \text{$w$ has odd number of $a$'s} \}
\end{align*}
then
\[
L = L' \cap L''
\]
Therefore we construct DFAs for $L'$ and $L''$ first.

The following $M'$ is a DFA such that $L(M') = L'$:
\begin{center}
\begin{tikzpicture}[shorten >=1pt,node distance=2cm,auto,initial text=]
\node[state,initial,accepting] (q_0)          {$q'_0$};
\node[state]                   (q_1) at (2,0) {$q'_1$};

\path[->] (q_0) edge [bend left]  node {$a,b$} (q_1)
          (q_1) edge [bend left]  node {$a,b$} (q_0)
;
\end{tikzpicture}
\end{center}

The following $M''$ is a DFA such that $L(M'') = L''$:
\begin{center}
\begin{tikzpicture}[shorten >=1pt,node distance=2cm,auto,initial text=]
\node[state,initial]   (q_0)          {$q''_0$};
\node[state,accepting] (q_1) at (2,0) {$q''_1$};

\path[->] (q_0) edge [bend left]  node {$a$} (q_1)
          (q_0) edge [loop above] node {$b$} ()

          (q_1) edge [bend left]  node {$a$} (q_0)
          (q_1) edge [loop above] node {$b$} ()

;
\end{tikzpicture}
\end{center}

The following is a DFA $M$ such that $L(M) = L(M') \cap L(M'')$:
\begin{center}
\begin{tikzpicture}[shorten >=1pt,node distance=2cm,auto,initial text=]
\node[state,initial]   (q_00)          {$(q'_0, q''_0)$};
\node[state]           (q_10) at (6,0) {$(q'_1, q''_0)$};

\node[state,accepting] (q_01) at (0,-6) {$(q'_0, q''_1)$};
\node[state]           (q_11) at (6,-6) {$(q'_1, q''_1)$};


\path[->] 
(q_00) edge [bend left=10,pos=0.9] node {$a$} (q_11)
(q_00) edge [bend left=10] node {$b$} (q_10)

(q_10) edge [bend left=10,pos=0.1] node {$a$} (q_01)
(q_10) edge [bend left=10,below]  node {$b$} (q_00)

(q_01) edge [bend left=10, pos=0.1] node {$a$} (q_10)
(q_01) edge [bend right=10,below] node {$b$} (q_11)

(q_11) edge [bend right=10,above] node {$b$} (q_01)
(q_11) edge [bend left=10,pos=0.9] node {$a$} (q_00)
          
; 
\end{tikzpicture}
\end{center}


\newpage

\end{document}
