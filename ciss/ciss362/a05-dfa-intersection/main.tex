\input{myassignmentpreamble}
\input{yliow}
\input{ciss362}
\renewcommand\TITLE{Assignment 5}
\renewcommand\AUTHOR{John Doe}
\renewcommand\EMAIL{jdoe1@jdoe1.com}

\begin{document}
\topmatter

We'll take a break from proofs and do some
DFA construction.

The questions are taken from our textbook.
When I say \lq\lq Textbook Q 1.1'' 
I mean exercise/problem 1.1 in our textbook.

Q1 and Q2 are not graded since answers are provided in the textbook.
Q3 is not graded because it's too easy -- I'm giving you the solution.

For Q4, (a), (b), (d), and (g) will not be graded.
I'm giving you the solution for 4(a) and (g).
The textbook provides solutions for Q4(b) and Q4(d).
Make sure you study my solutions and the author's solutions carefully.

Also, as I mentioned at the beginning of this course, I might not grade
all questions.
However you are strongly encouraged to solve every 
single problems.




\newpage
Q1. Textbook Q 1.1. (Not graded.)

\SOLUTION
\begin{align*}
           1 a &= a                                     & & \text{by M4} \\
\THEREFORE 1 c &= a \text{ for some integer $c (= a)$ } & & \\
\THEREFORE 1   &\mid a                                  & & \text{by D2}
\end{align*}

{\bf Note.}
The above shows you how to take a fact and produce another that involves
\lq\lq for some ....''.
Basically if you have a propositional formula 
$P(x)$ where $x$ is a variable, 
then if $P(v)$ is true where $v$ is a value, then
\lq\lq $P(x)$ is true for some value for x'' must also be true (duh).
It's pretty obvious right?

It's the same as saying if 
\[
\text{\lq\lq I have a pebble in my pocket''}
\]
then of course 
\[
\text{ \lq\lq I have an $x$ in my pocket for some $x$. }
\]
Right?
This is an \lq\lq axiom'' or rule in logic meaning to say
that this way of deducing a new fact is allowed because it models
the way human beings think.
Because this axiom produces a new fact, it's also
called an {\bf inference rule}.

Note that the \lq\lq opposite'' of that is not true!
Just because I can say that \lq\lq 
I have an $x$ in my pocket
 for some $x$'',
it does not mean that \lq\lq I have a pebble in my pocket''
because what I have in my pocket might very well be my pet lizard.

This is basically what you see in your discrete math class
as an axiom in logic:
\begin{align*}
           &P(a) \\
\THEREFORE &\exists x (P(x))
\end{align*}
This inference rule is called existential generalization.
From now on we'll call it EG.
So you should write the proof like this:
\begin{align*}
           1 a &= a                                     & & \text{by M4} \\
\THEREFORE 1 c &= a \text{ for some integer $c (= a)$ } & & \text{by EG} \\
\THEREFORE 1   &\mid a                                  & & \text{by D2}
\end{align*}

{\bf Note.}
Note that the only reason why proofs at an undergraduate level are
written so tediously is because you have to learn how to think and
argue logically and precisely.
The above format allows you to check the correctness of your logic.
Papers written even in research journals are actually {\it not} 
written in the above format. 
For instance in a paper one would write:
\begin{enumerate}
\item[]
{\it Since 1a = a, we have $1c = a$ for some integer $c$
and hence by definition $1 \mid a$, i.e. $1$ divides $a$.}
\end{enumerate}
or even
\begin{enumerate}
\item[] {\it Since 1a = a, by definition $1$ divides $a$.}
\end{enumerate}

{\bf Note.}
The application of \lq\lq rule'' D1 or D2 is not a deduction
(or inference).
It's just a linguistic translation of notation and definition.





\newpage
Q2. Textbook Q 1.2. (Not graded.)

\SOLUTION
Before we prove the above statement we will prove the following Lemma that we 
use later:

{\bf Lemma.}
Let $\delta: Q \times \Sigma \rightarrow Q$ be any transition function and 
$\delta^*: Q \times \Sigma^* \rightarrow Q$ be the associated function.
Then if $q \in Q$ and $c \in \Sigma$ 
\[
\delta^*(q, c) = \delta(q, c)
\]


{\it Proof of Lemma.} 
Let $c \in \Sigma$ (i.e. $c \in \Sigma^*$ of length 1).
We have the following:
\begin{align*}
\delta^*(q, c)
&= \delta^*(q, c\ep)           & & \text{by C7}\\
&= \delta^*(\delta(q,c), \ep)            & & \text{by DS3} \\
&= \delta(q, c)                & & \text{by DS1}
\end{align*}
The Lemma is proved.
QED.


We will prove the above statement by Mathematical Induction.
For $n \geq 1$, we define
\[
P(n):
\text{
If $x\in \Sigma^*$ with $|x| =n$, then ...
}
\]

\underline{Base case.}
\begin{align*}
|x| = 0                                                     & & \\
x = \ep                                                     & & \text{by L5, $\alpha$}\\
\delta^*((q',q''),x)=\delta^*((q',q''),\ep)& & \text{by $\alpha$} \\
\delta^*((q',q''),\ep)=(q',q'')                             & & \text{by DS1}\\
(q',q'')=(\delta'^*(q',\ep),q)                              & & \text{by DS2}\\
(\delta^*(q',\ep),q)=(\delta'^*(q',\ep),\delta''^*(q'',\ep))& & \text{by DS2}\\
(\delta'^*(q',\ep),\delta''^*)(q'',\ep))=(\delta'^*(q',x),\delta''^*(q'',x))                                                & & \text{by $\alpha$}
\end{align*}


\underline{Inductive case.} 
\begin{align*}
|x| = n + 1                                                                                                     & & \\
|x|\ne 0,x\ne\ep                                                                                                & & \text{by L6} \\
x=x'x''\text{ for some }x'\in\Sigma,\text{ }x''\in\Sigma^*                                                      & & \text{by C8, $\emptyset$} \\
|x'|=1                                                                                                          & & \text{by L2} \\
|x''| = n                                                                                                       & & \text{by L2} \\
\delta^*(q'q'',x)=\delta^*(q'q'',x'x'')                                                                         & & \text{by $\emptyset$} \\
\delta^*(q'q'',x'x'')=(\delta^*(\delta^*(q'q''),x'),x'')                                                        & & \text{by Lemma} \\
(\delta^*(\delta^*(q'q''),x'),x'')=(\delta^*(\delta'(q',x')),\delta''(q'',x')),x'')                             & & \text{by definition of $\delta$} \\
(\delta^*(\delta'(q',x')),\delta''(q'',x')),x'')=(\delta'^*(\delta'(q',x'),x''),\delta''^*(\delta''(q'',x'),x'')) & & \text{by Assumption} \\
(\delta'^*(\delta'(q',x'),x''),\delta''^*(\delta''(q'',x'),x''))=(\delta'^*(q',x'x''),\delta''^*(q'',x'x''))    & & \text{by DS4} \\
(\delta'^*(q',x'x''),\delta''^*(q'',x'x''))=(\delta'^*(q',x),\delta''^*(q'',x))                                 & & \text{by $\emptyset$}
\end{align*}

Hence by Mathematical Induction for all $n \geq 1$
\[
P(n+1)\text{ is true.}
\]
We conclude that if $x \in \Sigma^*$, then
\[
(\delta^*((q',q''),x))=(\delta'^*(q',x),\delta''^*(q'',x))  
\]




\newpage
Q3. Textbook Q 1.3

\SOLUTION
\begin{align*}
a &\mid b \\
\THEREFORE ax &= b \text{ for some integer $x$}       & & \text{by D1} \\
\THEREFORE (ax)c &= bc \text{ for some integer $x$}   & & \text{by F1} \\
\THEREFORE a(xc) &= bc \text{ for some integer $x$}   & & \text{by M2} \\
\THEREFORE a(xc) &= bc \text{ for some integer $xc$}  & & \text{by M1} \\
\THEREFORE ay &= bc \text{ for some integer $y(=xc)$} & & \text{by EG} \\
\THEREFORE a &\mid bc                                 & & \text{by D2} \\
\end{align*}




\newpage
Q4. Textbook Q1.4. ((b), (d), and (g) not graded.)

Each language in this question can be written as an intersection
of two languages that can accepted by DFAs.
Given two DFAs, say $M_1$ and $M_2$, 
then it's always possible to design another DFA $M$ such that
\[
L(M) = L(M_1) \cap L(M_2).
\]
See the textbook.

Try to layout the states in the \lq\lq obvious'' way.
(Look at the solution for (a) and (g).)

\SOLUTION


4(a)
Let 
\[
L = \{w \mid \text{$w$ has at least three $a$'s and at least two $b$'s}\}
\]
Note that if 
\begin{align*}
L' &= \{w \mid \text{$w$ has at least three $a$'s} \} \\
L'' &= \{w \mid \text{$w$ has at least two $b$'s} \}
\end{align*}
then
\[
L = L' \cap L''
\]
Therefore we construct DFAs for $L'$ and $L''$ first.

The following $M'$ is a DFA such that $L(M') = L'$:
\begin{center}
\begin{tikzpicture}[shorten >=1pt,node distance=2cm,auto,initial text=]
\node[state,initial]   (q_0)          {$q'_0$};
\node[state]           (q_1) at (2,0) {$q'_1$};
\node[state]           (q_2) at (4,0) {$q'_2$};
\node[state,accepting] (q_3) at (6,0) {$q'_3$};

\path[->] (q_0) edge node {$a$} (q_1)
          (q_0) edge [loop above] node {$b$} ()
          (q_1) edge node {$a$} (q_2)
          (q_1) edge [loop above] node {$b$} ()
          (q_2) edge node {$a$} (q_3)
          (q_2) edge [loop above] node {$b$} ()
          (q_3) edge [loop above] node {$a, b$} ()
;
\end{tikzpicture}
\end{center}
The following $M''$ is a DFA such that $L(M'') = L''$:
\begin{center}
\begin{tikzpicture}[shorten >=1pt,node distance=2cm,auto,initial text=]
\node[state,initial]   (q_0)          {$q''_0$};
\node[state]           (q_1) at (2,0) {$q''_1$};
\node[state,accepting] (q_2) at (4,0) {$q''_2$};

\path[->] (q_0) edge node {$b$} (q_1)
          (q_0) edge [loop above] node {$a$} ()
          (q_1) edge node {$b$} (q_2)
          (q_1) edge [loop above] node {$a$} ()
          (q_2) edge [loop above] node {$a, b$} ()
;
\end{tikzpicture}
\end{center}
The following is a DFA $M$ such that $L(M) = L(M') \cap L(M'')$:
\begin{center}
\begin{tikzpicture}[shorten >=1pt,node distance=2cm,auto,initial text=]
\node[state,initial]   (q_00)          {$(q'_0, q''_0)$};
\node[state]           (q_10) at (3,0) {$(q'_1, q''_0)$};
\node[state]           (q_20) at (6,0) {$(q'_2, q''_0)$};
\node[state]           (q_30) at (9,0) {$(q'_3, q''_0)$};
\node[state]           (q_01) at (0,-4) {$(q'_0, q''_1)$};
\node[state]           (q_11) at (3,-4) {$(q'_1, q''_1)$};
\node[state]           (q_21) at (6,-4) {$(q'_2, q''_1)$};
\node[state]           (q_31) at (9,-4) {$(q'_3, q''_1)$};
\node[state]           (q_02) at (0,-8) {$(q'_0, q''_2)$};
\node[state]           (q_12) at (3,-8) {$(q'_1, q''_2)$};
\node[state]           (q_22) at (6,-8) {$(q'_2, q''_2)$};
\node[state,accepting] (q_32) at (9,-8) {$(q'_3, q''_2)$};

\path[->] (q_00) edge node {$a$} (q_10)
          (q_00) edge node {$b$} (q_01)
          (q_10) edge node {$a$} (q_20)
          (q_10) edge node {$b$} (q_11)
          (q_20) edge node {$a$} (q_30)
          (q_20) edge node {$b$} (q_21)
          (q_30) edge [loop right] node {$a$} ()
          (q_30) edge node {$b$} (q_31)

          (q_01) edge node {$a$} (q_11)
          (q_01) edge node {$b$} (q_02)
          (q_11) edge node {$a$} (q_21)
          (q_11) edge node {$b$} (q_12)
          (q_21) edge node {$a$} (q_31)
          (q_21) edge node {$b$} (q_22)
          (q_31) edge [loop right] node {$a$} ()
          (q_31) edge node {$b$} (q_32)

          (q_02) edge node {$a$} (q_12)
          (q_02) edge [loop below] node {$b$} ()
          (q_12) edge node {$a$} (q_22)
          (q_12) edge [loop below] node {$b$} ()
          (q_22) edge node {$a$} (q_32)
          (q_22) edge [loop below] node {$b$} ()
          (q_32) edge [loop right] node {$a, b$} ()

;
\end{tikzpicture}
\end{center}


\newpage
\input{q4b.tex}
\newpage
\input{q4c.tex}
\newpage
\input{q4d.tex}
\newpage
\input{q4e.tex}
\newpage
\input{q4f.tex}
\newpage

4(g)
Let 
\[
L = \{w \mid \text{$w$ has even length and odd number of $a$'s}\}
\]
Note that if 
\begin{align*}
L' &= \{w \mid \text{$w$ has even length} \} \\
L'' &= \{w \mid \text{$w$ has odd number of $a$'s} \}
\end{align*}
then
\[
L = L' \cap L''
\]
Therefore we construct DFAs for $L'$ and $L''$ first.

The following $M'$ is a DFA such that $L(M') = L'$:
\begin{center}
\begin{tikzpicture}[shorten >=1pt,node distance=2cm,auto,initial text=]
\node[state,initial,accepting] (q_0)          {$q'_0$};
\node[state]                   (q_1) at (2,0) {$q'_1$};

\path[->] (q_0) edge [bend left]  node {$a,b$} (q_1)
          (q_1) edge [bend left]  node {$a,b$} (q_0)
;
\end{tikzpicture}
\end{center}

The following $M''$ is a DFA such that $L(M'') = L''$:
\begin{center}
\begin{tikzpicture}[shorten >=1pt,node distance=2cm,auto,initial text=]
\node[state,initial]   (q_0)          {$q''_0$};
\node[state,accepting] (q_1) at (2,0) {$q''_1$};

\path[->] (q_0) edge [bend left]  node {$a$} (q_1)
          (q_0) edge [loop above] node {$b$} ()

          (q_1) edge [bend left]  node {$a$} (q_0)
          (q_1) edge [loop above] node {$b$} ()

;
\end{tikzpicture}
\end{center}

The following is a DFA $M$ such that $L(M) = L(M') \cap L(M'')$:
\begin{center}
\begin{tikzpicture}[shorten >=1pt,node distance=2cm,auto,initial text=]
\node[state,initial]   (q_00)          {$(q'_0, q''_0)$};
\node[state]           (q_10) at (6,0) {$(q'_1, q''_0)$};

\node[state,accepting] (q_01) at (0,-6) {$(q'_0, q''_1)$};
\node[state]           (q_11) at (6,-6) {$(q'_1, q''_1)$};


\path[->] 
(q_00) edge [bend left=10,pos=0.9] node {$a$} (q_11)
(q_00) edge [bend left=10] node {$b$} (q_10)

(q_10) edge [bend left=10,pos=0.1] node {$a$} (q_01)
(q_10) edge [bend left=10,below]  node {$b$} (q_00)

(q_01) edge [bend left=10, pos=0.1] node {$a$} (q_10)
(q_01) edge [bend right=10,below] node {$b$} (q_11)

(q_11) edge [bend right=10,above] node {$b$} (q_01)
(q_11) edge [bend left=10,pos=0.9] node {$a$} (q_00)
          
; 
\end{tikzpicture}
\end{center}


\newpage

\end{document}
