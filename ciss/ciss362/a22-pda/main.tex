\input{myassignmentpreamble}
\input{yliow}
\input{ciss362}
\renewcommand\TITLE{Assignment 22}
\renewcommand\AUTHOR{John Doe}
\renewcommand\EMAIL{jdoe@jdoe.com}

\begin{document}
\topmatter


The transition of a PDA looks like this:
\[
(a, b \rightarrow c)
\]
which means: \lq\lq read $a$, replace $b$ on the top of the stack with 
$c$, and go to the next state''.
Note that $a$ or $b$ or $c$ can be $\ep$.
For instance this is a valid transition:
\[
(\ep, \ep \rightarrow \ep)
\]
which means \lq\lq don't any input, don't change the stack, go to the 
next state.''
The transition
\[
(a, \ep \rightarrow b)
\]
means \lq\lq  read $a$, put $b$ on top of the stack, and go to the next state
''.
The transition
\[
(a, b \rightarrow \ep)
\]
means \lq\lq  read $a$, remove $b$ from the top of the stack, and go to the 
next state''.
Etc.


The characters used in the stack can be different from the 
input characters.
For instance you have already seen that we can use the \$ character to mark
the bottom of the stack.
If $\Sigma$ denotes the input characters and $\Gamma$ denotes the 
characters used in the stack, then
a transition looks like this
\[
(a, b \rightarrow c)
\]
where $a \in \Sigma \cup \{\ep\}$ and $b,c \in \Gamma \cup \{\ep\}$.

\newpage 

Q1. Construct, if possible, a PDA that accepts
\[
L = \{a^m b^n \mid m \neq n\}
\]

[HINT: Rewrite the language as a union of two, design two PDAs, and
finally construct a PDA that accepts the union.]

\SOLUTION

\begin{align*}
           1 a &= a                                     & & \text{by M4} \\
\THEREFORE 1 c &= a \text{ for some integer $c (= a)$ } & & \\
\THEREFORE 1   &\mid a                                  & & \text{by D2}
\end{align*}

{\bf Note.}
The above shows you how to take a fact and produce another that involves
\lq\lq for some ....''.
Basically if you have a propositional formula 
$P(x)$ where $x$ is a variable, 
then if $P(v)$ is true where $v$ is a value, then
\lq\lq $P(x)$ is true for some value for x'' must also be true (duh).
It's pretty obvious right?

It's the same as saying if 
\[
\text{\lq\lq I have a pebble in my pocket''}
\]
then of course 
\[
\text{ \lq\lq I have an $x$ in my pocket for some $x$. }
\]
Right?
This is an \lq\lq axiom'' or rule in logic meaning to say
that this way of deducing a new fact is allowed because it models
the way human beings think.
Because this axiom produces a new fact, it's also
called an {\bf inference rule}.

Note that the \lq\lq opposite'' of that is not true!
Just because I can say that \lq\lq 
I have an $x$ in my pocket
 for some $x$'',
it does not mean that \lq\lq I have a pebble in my pocket''
because what I have in my pocket might very well be my pet lizard.

This is basically what you see in your discrete math class
as an axiom in logic:
\begin{align*}
           &P(a) \\
\THEREFORE &\exists x (P(x))
\end{align*}
This inference rule is called existential generalization.
From now on we'll call it EG.
So you should write the proof like this:
\begin{align*}
           1 a &= a                                     & & \text{by M4} \\
\THEREFORE 1 c &= a \text{ for some integer $c (= a)$ } & & \text{by EG} \\
\THEREFORE 1   &\mid a                                  & & \text{by D2}
\end{align*}

{\bf Note.}
Note that the only reason why proofs at an undergraduate level are
written so tediously is because you have to learn how to think and
argue logically and precisely.
The above format allows you to check the correctness of your logic.
Papers written even in research journals are actually {\it not} 
written in the above format. 
For instance in a paper one would write:
\begin{enumerate}
\item[]
{\it Since 1a = a, we have $1c = a$ for some integer $c$
and hence by definition $1 \mid a$, i.e. $1$ divides $a$.}
\end{enumerate}
or even
\begin{enumerate}
\item[] {\it Since 1a = a, by definition $1$ divides $a$.}
\end{enumerate}

{\bf Note.}
The application of \lq\lq rule'' D1 or D2 is not a deduction
(or inference).
It's just a linguistic translation of notation and definition.
\newpage



Q2. Construct, if possible, a PDA that accepts
\[
L = \{a^m b^m a^n b^n \mid m \geq 0, n \geq 0\}
\]

[HINT: Rewrite the language as a concatenation of two,
design two PDAs, finally construct a PDA for the concatenation.]


\SOLUTION

Before we prove the above statement we will prove the following Lemma that we 
use later:

{\bf Lemma.}
Let $\delta: Q \times \Sigma \rightarrow Q$ be any transition function and 
$\delta^*: Q \times \Sigma^* \rightarrow Q$ be the associated function.
Then if $q \in Q$ and $c \in \Sigma$ 
\[
\delta^*(q, c) = \delta(q, c)
\]


{\it Proof of Lemma.} 
Let $c \in \Sigma$ (i.e. $c \in \Sigma^*$ of length 1).
We have the following:
\begin{align*}
\delta^*(q, c)
&= \delta^*(q, c\ep)           & & \text{by C7}\\
&= \delta^*(\delta(q,c), \ep)            & & \text{by DS3} \\
&= \delta(q, c)                & & \text{by DS1}
\end{align*}
The Lemma is proved.
QED.


We will prove the above statement by Mathematical Induction.
For $n \geq 1$, we define
\[
P(n):
\text{
If $x\in \Sigma^*$ with $|x| =n$, then ...
}
\]

\underline{Base case.}
\begin{align*}
|x| = 0                                                     & & \\
x = \ep                                                     & & \text{by L5, $\alpha$}\\
\delta^*((q',q''),x)=\delta^*((q',q''),\ep)& & \text{by $\alpha$} \\
\delta^*((q',q''),\ep)=(q',q'')                             & & \text{by DS1}\\
(q',q'')=(\delta'^*(q',\ep),q)                              & & \text{by DS2}\\
(\delta^*(q',\ep),q)=(\delta'^*(q',\ep),\delta''^*(q'',\ep))& & \text{by DS2}\\
(\delta'^*(q',\ep),\delta''^*)(q'',\ep))=(\delta'^*(q',x),\delta''^*(q'',x))                                                & & \text{by $\alpha$}
\end{align*}


\underline{Inductive case.} 
\begin{align*}
|x| = n + 1                                                                                                     & & \\
|x|\ne 0,x\ne\ep                                                                                                & & \text{by L6} \\
x=x'x''\text{ for some }x'\in\Sigma,\text{ }x''\in\Sigma^*                                                      & & \text{by C8, $\emptyset$} \\
|x'|=1                                                                                                          & & \text{by L2} \\
|x''| = n                                                                                                       & & \text{by L2} \\
\delta^*(q'q'',x)=\delta^*(q'q'',x'x'')                                                                         & & \text{by $\emptyset$} \\
\delta^*(q'q'',x'x'')=(\delta^*(\delta^*(q'q''),x'),x'')                                                        & & \text{by Lemma} \\
(\delta^*(\delta^*(q'q''),x'),x'')=(\delta^*(\delta'(q',x')),\delta''(q'',x')),x'')                             & & \text{by definition of $\delta$} \\
(\delta^*(\delta'(q',x')),\delta''(q'',x')),x'')=(\delta'^*(\delta'(q',x'),x''),\delta''^*(\delta''(q'',x'),x'')) & & \text{by Assumption} \\
(\delta'^*(\delta'(q',x'),x''),\delta''^*(\delta''(q'',x'),x''))=(\delta'^*(q',x'x''),\delta''^*(q'',x'x''))    & & \text{by DS4} \\
(\delta'^*(q',x'x''),\delta''^*(q'',x'x''))=(\delta'^*(q',x),\delta''^*(q'',x))                                 & & \text{by $\emptyset$}
\end{align*}

Hence by Mathematical Induction for all $n \geq 1$
\[
P(n+1)\text{ is true.}
\]
We conclude that if $x \in \Sigma^*$, then
\[
(\delta^*((q',q''),x))=(\delta'^*(q',x),\delta''^*(q'',x))  
\]


\newpage






\end{document}
