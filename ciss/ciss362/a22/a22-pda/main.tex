\input{myassignmentpreamble}
\input{yliow}
\input{ciss362}
\renewcommand\TITLE{Assignment 22}
\renewcommand\AUTHOR{John Doe}
\renewcommand\EMAIL{jdoe@jdoe.com}

\begin{document}
\topmatter


The transition of a PDA looks like this:
\[
(a, b \rightarrow c)
\]
which means: \lq\lq read $a$, replace $b$ on the top of the stack with 
$c$, and go to the next state''.
Note that $a$ or $b$ or $c$ can be $\ep$.
For instance this is a valid transition:
\[
(\ep, \ep \rightarrow \ep)
\]
which means \lq\lq don't any input, don't change the stack, go to the 
next state.''
The transition
\[
(a, \ep \rightarrow b)
\]
means \lq\lq  read $a$, put $b$ on top of the stack, and go to the next state
''.
The transition
\[
(a, b \rightarrow \ep)
\]
means \lq\lq  read $a$, remove $b$ from the top of the stack, and go to the 
next state''.
Etc.


The characters used in the stack can be different from the 
input characters.
For instance you have already seen that we can use the \$ character to mark
the bottom of the stack.
If $\Sigma$ denotes the input characters and $\Gamma$ denotes the 
characters used in the stack, then
a transition looks like this
\[
(a, b \rightarrow c)
\]
where $a \in \Sigma \cup \{\ep\}$ and $b,c \in \Gamma \cup \{\ep\}$.

\newpage 

Q1. Construct, if possible, a PDA that accepts
\[
L = \{a^m b^n \mid m \neq n\}
\]

[HINT: Rewrite the language as a union of two, design two PDAs, and
finally construct a PDA that accepts the union.]

\SOLUTION



\newpage



Q2. Construct, if possible, a PDA that accepts
\[
L = \{a^m b^m a^n b^n \mid m \geq 0, n \geq 0\}
\]

[HINT: Rewrite the language as a concatenation of two,
design two PDAs, finally construct a PDA for the concatenation.]


\SOLUTION




\newpage






\end{document}
