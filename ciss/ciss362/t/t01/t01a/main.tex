\input{myassignmentpreamble}
\input{yliow}
\input{ciss362}

\renewcommand\TITLE{Test 1 Part A}

\renewcommand\AUTHOR{John Doe}
\renewcommand\EMAIL{jdoe@jdoe.com}
\newcommand\tf{T or F or M: }

\begin{document}
\topmatter

The following instructions on defining a DFA or an NFA
must be followed.

Here's an example on how to define an NFA:
\begin{console}
automata:nfa
sigma:a,b
states:q0,q1,q2,q3,q4
start:q0
accept:q0,q1
transitions:
q0,a,q0
q0,b,q1
q1,e,q3
\end{console}
The letter \verb!e! is used for $\epsilon$.
(None of the $\Sigma$ in this test will use \verb!e!.)

Here's an example on how to define a DFA:
\begin{console}
automata:dfa
sigma:a,b
states:q0,q1
start:q0
accept:q1
transitions:
q0,a,q0
q0,b,q1
q1,a,q1
q1,b,q0
\end{console}




\newpage

Q1. Our alphabet is $\Sigma = \{a, b, c\}$.

\begin{enumerate}

\item \tf 1 + 1 = 2

\item \tf $a$ is a regular expression

\item \tf $a\cup b$ is a regular expression

\item \tf $a \cdot \cup c$ is a regular expression

\item \tf $a \cup^* b$ is a regular expression

\item \tf $\{c\}$ is a regular expression

\item \tf $c \cdot \emptyset $ is a regular expression

\item \tf $\ep \cdot \ep \cdot \ep$ is a regular expression

\item \tf $\emptyset^*$ is a regular expression

\item \tf $a^*)$ is a regular expression

\item \tf $a\cdot b \cup c$ is a regular expression

\item \tf $a^b$ is a regular expression

\item \tf $a \in L(a \cup b)$

\item \tf $ab \in L(a^* \cup b^*)$

\item \tf $ab \in L((a \cup b)^*)$

\item \tf $a \in L(a \cdot \emptyset)$

\item \tf $ab \in L(a \cdot (a \cup b) \cdot c^*)$

\item \tf $ab \in L((a \cup b) \cdot (b \cup c))$

\item \tf $ab \in L((a \cup \overline{b}))$

\item \tf $a^4 b^2 \in L(a^* \cup b) L(a \cup b^*)$

\end{enumerate}

SOLUTION ON NEXT PAGE ...

\newpage

\SOLUTION

Modify the file \verb!q01.tex!.
Use the letter \verb!t! or \verb!f! or \verb!m!.
I have already completed the first question for you.

\begin{console}
1:t
2:t
3:t
4:m
5:m
6:f
7:t
8:t
9:t
10:m
11:t
12:m
13:t
14:f
15:t
16:f
17:t
18:t
19:m
20:t
\end{console}




\newpage

Q2.
For the NFA $N$ given below, 
using the subset construction, construct a DFA $M$ that accepts
the same language accepted by $N$. 
Do not include states which are not 
reachable from the initial state of your DFA.

\begin{center}
\begin{tikzpicture}[shorten >=1pt,node distance=2cm,auto,initial text=]
\node[state,initial] (A) at (  0,  0) {$q_0$};
\node[state,accepting] (C) at (  4,  0) {$q_2$};
\node[state] (B) at (  2,  0) {$q_1$};
\node[state] (E) at (  8,  0) {$q_4$};
\node[state,accepting] (D) at (  6,  0) {$q_3$};

\path[->]
(A) edge [loop above] node {$b$} ()
(A) edge [bend left=0,pos=0.5,above] node {$\ep$} (B)
(B) edge [bend left=0,pos=0.5,above] node {$a,b$} (C)
(B) edge [bend left=40,pos=0.5,above] node {$a$} (D)
(D) edge [bend left=40,pos=0.5,below] node {$b$} (B)
(C) edge [bend left=35,pos=0.5,above] node {$a$} (E)
(C) edge [bend left=0,pos=0.5,above] node {$\ep$} (D)
(D) edge [bend left=10,pos=0.5,above] node {$\ep$} (E)
(E) edge [bend left=10,pos=0.5,below] node {$b$} (D)

;
\end{tikzpicture}
\end{center}

\SOLUTION

Modify the file \verb!q02.tex!.

\begin{console}
automata:


























\end{console}





\newpage

Q3. Design an NFA that accepts $\{a, ab, bab\}^*$.

\SOLUTION

Modify the file \verb!q03.tex!.

\begin{console}
automata:















\end{console}






\newpage

Q4. Recall that the \lq\lq complement construction''
works for a DFA, i.e., if you exchange
\[
\text{accept $\leftrightarrow$ non-accept states}
\]
the resulting DFA will accept the complement of the language
accepting by the original DFA.

Does it work with NFAs?
In other words, if you exchange 
\[
\text{accept $\leftrightarrow$ non-accept states}
\]
for an NFA,
will the 
resulting NFA accept the complement of the language
accepting by the original NFA?
If it works, prove it.
If it does not, provide a minimal counterexample.
(Minimal in this case means the one with least number of 
states.)

\SOLUTION

Modify \verb!q04.tex!.

\begin{console}
False.

Let NFA accept nothing
automata: nfa
sigma: a, b
states: q0
start: q0
accept:
transition:

Then, the compliment of this NFA accepts the empty string (e)
automata: nfa
sigma: a, b
states: q0
start: q0
accept: q0
transition:

The DFA for the original language is
automata: dfa
sigma: a, b
states: q0
start: q0
accept:
transition:
q0, a, q0
q0, b, q0

The compliment of the DFA accepts a*b*
automata: dfa
sigma: a, b
states: q0
start: q0
accept: a0
transition:
q0, a, q0
q0, b, q0

Then the compliment of the NFA doesn't equal the compliment of the DFA,
and therefore this operation isn't the same for NFA as DFA
\end{console}


\end{document}
