

3(b)
Let $L = \{0^m 1^n \mid m \neq n \}$.

We will prove this by contradiction.
Assume that $L$ is regular.

We already know that complementation is a regular operator,
i.e. if $L'$ is a regular language, then
$\overline{L'}$ is also regular.
Since, according to our assumption, $L$ is regular,
\[
\overline{L} = \{x \in \{0,1\}^* \mid x \text{ is not of the form } 0^m 1^n
\text{ for } m \neq n \}
\] 
is also regular
Note that $\{0^n 1^n \mid n \geq 0\} \subseteq \overline{L}$.
(WARNING: $\{0^n 1^n \mid n \geq 0\} \neq \overline{L}$
since for instance $1010 \in \overline{L}$.)

Note that intersection is a regular operator, i.e.
if $L', L''$ are regular, then $L' \cup L''$ is also regular.
$L(0^*1^*)$ is regular since it is the language accepted by the
regular expression $0^* 1^*$.
Therefore, since from the above $\overline{L}$ is regular,
\[
\overline{L} \underline{\cap} L(0^* 1^*) = \{0^n 1^n \mid n \geq 0\}
\]
is also regular.

This is a contradiction since we have already proven that 
$\{0^n 1^n \mid n \geq 0\}$
is not regular.
Hence our assumption that $L = \{0^m 1^n \mid m \neq n \}$
is regular cannot hold.
Therefore $L$ must be non-regular.
