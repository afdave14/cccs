%\newcommand\COURSENAME{Theory of Automata, Languages, and Computation}
%\newcommand\COURSESHORTNAME{Automata}
%\newcommand\COURSENUMBER{CISS362/MATH362}

\input{myassignmentpreamble}
\input{yliow}
\input{ciss362}
\renewcommand\TITLE{Assignment 20}
\renewcommand\AUTHOR{David Campbell}
\renewcommand\EMAIL{djcampbell2@cougars.ccis.edu}

\begin{document}
\topmatter


The following is the version of the lemma that you must use:


{\bf Pumping Lemma (for regular language).}
Let $L$ be a language (over $\Sigma$).
Suppose that for each integer $n \geq 0$, there is some 
$x \in L$ such that $|x| \geq n$ and, 
if $x = uvw$ where $u,v,w \in \Sigma^*$, then
there is some integer $i_0 \geq 0$ such that
\[
uv^{i_0}w \not\in L
\]
Then $L$ is not regular. \qed

I expect all proofs to be properly written.
This means that not only must the proof be logically correct,
it must for instance be grammatically correct.
Proper mathematical notation must be used.
Make sure you check the spelling of every word.

\newpage


Q1. Textbook Q 1.29 and (d) below.

(d) $A_4 = \{ a^{42^n} \mid n \geq 0 \}$

[The answers for (a) and (c) are already given in the book.
Make sure you study it.
You only need to complete (b) and (d).]

\SOLUTION


1(a)
Let $n \geq 0$.
We choose $x = 0^n 1^n 2^n$
Note that $x \in A_1$ and $|x| = n + n + n = 3n \geq n$.

Let $x = uvw$ where
\[
|uv| \leq n, \,\,\,\,\, |v| > 0
\]
Note that since $uv$ is a prefix of $x = 0^n 1^n 2^n$ and $|uv| \leq n$,
the only characters appearing in $uv$ is $0$.
Hence
\[
u = 0^{|u|}, \,\,\,\,\,
v = 0^{|v|}, \,\,\,\,\,
w = 0^{n - |u| - |v|} 1^n 2^n
\]
Therefore if we choose $i_0 = 2$, we obtain
\begin{align*}
u v^{i_0} w 
&= u v^2 w \\
&= 0^{|u|} \cdot 0^{2|v|} \cdot 0^{n - |u| - |v|} 1^n 2^n \\
&= 0^{n + |v|} 1^n 2^n
\end{align*}
Since $|v| > 0$, $n + |v| \neq n$.
Hence $u v^{i_0} w \not\in A_1$.

By the pumping lemma for regular language,
we conclude that $A_1$ is not regular.
 \newpage

\input{q1b.tex} \newpage


1(c)
Let $n \geq 0$.
We choose $x = a^{2^n}$.
Note that $x \in A_3$ and $|x| = 2^n \geq n$.

Let $x = uvw$ with $|uv| \leq n$ and $|v| > 0$.
We want to choose $i_0$ such that
\[
u v^{i_0} w \not\in A_3
\]

[The strategy that we hope will work is to
show that $2^n < |uv^{i_0}w| < 2^{n+1}$.]

Note that 
\[
|u v^{i_0} w| = 2^n + (i_0 - 1)|v|
\]
Since $|v| > 0$ if we choose $i_0$ such that
\[
i_0 - 1 \geq 1
\]
then we must have
\[
2^n < 2^n + (i_0 - 1) |v|
\]
We also have
\[
|v| \leq |uv| 
\]
Therefore
\[
2^n < 2^n + (i_0 - 1) |v| \leq 2^n + (i_0 - 1) |uv| 
\]
We also know that $|uv| \leq n$.
Therefore
\[
2^n < 2^n + (i_0 - 1) |v| \leq 2^n + (i_0 - 1) |uv| \leq 2^n + (i_0 - 1) n
\]
Therefore if we choose $i_0$ such that
\[
(i_0 - 1) n < 2^n
\]
then we have
\[
2^n < 2^n + (i_0 - 1) |v| \leq 2^n + (i_0 - 1) n < 2^n + 2^n = 2\cdot 2^n 
= 2^{n+1}
\]
Altogether we have to choose $i_0$ such that 
\begin{enumerate}
\item $i_0 - 1 \geq 1$
\item $(i_0 - 1) n < 2^n$
\end{enumerate}
If we choose $i_0 = 2$
both conditions are indeed
satisfied:
\begin{enumerate}
\item $i_0 - 1 = 2 - 1 \geq 1$
\item $(i_0 - 1) n = (2 - 1) n = n < 2^n$
\end{enumerate}
[The fact that $n < 2^n$ (for $n \geq 0$) can be shown easily.]
In that case we have
\[
2^n < 2^n + (i_0 - 1) |v| < 2^{n+1}
\]
and therefore $uv^{i_0}w \underline{\not \in} A_3$ 

By the pumping lemma for regular languages,
$A_3$ is not regular.
 \newpage

\input{q1d.tex} \newpage

\newpage




Q2. Textbook Q 1.30.
[This is optional and is not graded.]

Before we prove the above statement we will prove the following Lemma that we 
use later:

{\bf Lemma.}
Let $\delta: Q \times \Sigma \rightarrow Q$ be any transition function and 
$\delta^*: Q \times \Sigma^* \rightarrow Q$ be the associated function.
Then if $q \in Q$ and $c \in \Sigma$ 
\[
\delta^*(q, c) = \delta(q, c)
\]


{\it Proof of Lemma.} 
Let $c \in \Sigma$ (i.e. $c \in \Sigma^*$ of length 1).
We have the following:
\begin{align*}
\delta^*(q, c)
&= \delta^*(q, c\ep)           & & \text{by C7}\\
&= \delta^*(\delta(q,c), \ep)            & & \text{by DS3} \\
&= \delta(q, c)                & & \text{by DS1}
\end{align*}
The Lemma is proved.
QED.


We will prove the above statement by Mathematical Induction.
For $n \geq 1$, we define
\[
P(n):
\text{
If $x\in \Sigma^*$ with $|x| =n$, then ...
}
\]

\underline{Base case.}
\begin{align*}
|x| = 0                                                     & & \\
x = \ep                                                     & & \text{by L5, $\alpha$}\\
\delta^*((q',q''),x)=\delta^*((q',q''),\ep)& & \text{by $\alpha$} \\
\delta^*((q',q''),\ep)=(q',q'')                             & & \text{by DS1}\\
(q',q'')=(\delta'^*(q',\ep),q)                              & & \text{by DS2}\\
(\delta^*(q',\ep),q)=(\delta'^*(q',\ep),\delta''^*(q'',\ep))& & \text{by DS2}\\
(\delta'^*(q',\ep),\delta''^*)(q'',\ep))=(\delta'^*(q',x),\delta''^*(q'',x))                                                & & \text{by $\alpha$}
\end{align*}


\underline{Inductive case.} 
\begin{align*}
|x| = n + 1                                                                                                     & & \\
|x|\ne 0,x\ne\ep                                                                                                & & \text{by L6} \\
x=x'x''\text{ for some }x'\in\Sigma,\text{ }x''\in\Sigma^*                                                      & & \text{by C8, $\emptyset$} \\
|x'|=1                                                                                                          & & \text{by L2} \\
|x''| = n                                                                                                       & & \text{by L2} \\
\delta^*(q'q'',x)=\delta^*(q'q'',x'x'')                                                                         & & \text{by $\emptyset$} \\
\delta^*(q'q'',x'x'')=(\delta^*(\delta^*(q'q''),x'),x'')                                                        & & \text{by Lemma} \\
(\delta^*(\delta^*(q'q''),x'),x'')=(\delta^*(\delta'(q',x')),\delta''(q'',x')),x'')                             & & \text{by definition of $\delta$} \\
(\delta^*(\delta'(q',x')),\delta''(q'',x')),x'')=(\delta'^*(\delta'(q',x'),x''),\delta''^*(\delta''(q'',x'),x'')) & & \text{by Assumption} \\
(\delta'^*(\delta'(q',x'),x''),\delta''^*(\delta''(q'',x'),x''))=(\delta'^*(q',x'x''),\delta''^*(q'',x'x''))    & & \text{by DS4} \\
(\delta'^*(q',x'x''),\delta''^*(q'',x'x''))=(\delta'^*(q',x),\delta''^*(q'',x))                                 & & \text{by $\emptyset$}
\end{align*}

Hence by Mathematical Induction for all $n \geq 1$
\[
P(n+1)\text{ is true.}
\]
We conclude that if $x \in \Sigma^*$, then
\[
(\delta^*((q',q''),x))=(\delta'^*(q',x),\delta''^*(q'',x))  
\]

\newpage




Q3. Textbook Q 1.46.

[I've provided the solution for (b). Make sure you study it.]

\input{q3a.tex} \newpage


3(b)
Let $L = \{0^m 1^n \mid m \neq n \}$.

We will prove this by contradiction.
Assume that $L$ is regular.

We already know that complementation is a regular operator,
i.e. if $L'$ is a regular language, then
$\overline{L'}$ is also regular.
Since, according to our assumption, $L$ is regular,
\[
\overline{L} = \{x \in \{0,1\}^* \mid x \text{ is not of the form } 0^m 1^n
\text{ for } m \neq n \}
\] 
is also regular
Note that $\{0^n 1^n \mid n \geq 0\} \subseteq \overline{L}$.
(WARNING: $\{0^n 1^n \mid n \geq 0\} \neq \overline{L}$
since for instance $1010 \in \overline{L}$.)

Note that intersection is a regular operator, i.e.
if $L', L''$ are regular, then $L' \cup L''$ is also regular.
$L(0^*1^*)$ is regular since it is the language accepted by the
regular expression $0^* 1^*$.
Therefore, since from the above $\overline{L}$ is regular,
\[
\overline{L} \underline{\cap} L(0^* 1^*) = \{0^n 1^n \mid n \geq 0\}
\]
is also regular.

This is a contradiction since we have already proven that 
$\{0^n 1^n \mid n \geq 0\}$
is not regular.
Hence our assumption that $L = \{0^m 1^n \mid m \neq n \}$
is regular cannot hold.
Therefore $L$ must be non-regular.
 \newpage
\input{q3c.tex} \newpage
\input{q3d.tex} \newpage


Q4. Either prove 
\[
L = \{ab^nab^n \mid n \geq 0 \}
\]
is regular by 
\underline{constructing}
a DFA or NFA or regular \underline{expression}
 for $L$
or prove it is not regular using the pumping lemma for regular languages.

[HINT: It's \lq\lq intuitively'' clear that it's
not regular.
However intuition can be used as a guide, not as a proof.
WARNING: If you're using the pumping lemma, note that 
the $uv$ might contains two types of characters, both $a$ and $b$.]

{\bf SOLUTION.}

Let $x,y$ be words in $\Sigma^*$ with $|x| = n + 1$.

We have the following:
\begin{align*}
           |x| &= n + 1  & & \tag{A} \\ 
\THEREFORE |x| &\geq 1 \\
\THEREFORE |x| &\neq 0 \\
\THEREFORE x &\neq ? & & \text{by Q2} \\
\THEREFORE x &= x'x'' \text{ for some } x' \in \Sigma, x'' \in \Sigma^* 
           & & \text{by ?} 
           \tag{B} \\
\THEREFORE |x| &= 1 + |x''| \text{ for some } x'' \in \Sigma^*
           & & \text{by ?} \\
\THEREFORE n + 1 &= 1 + |x''| \text{ for some } x'' \in \Sigma^*
           & & \text{by A} \\
\THEREFORE |x''| &= n \text{ for some } x'' \in \Sigma^*
\end{align*}

Therefore for some $x' \in \Sigma$ and $x'' \in \Sigma^*$
we have the following:
\begin{align*}
(xy)^R
&= (?x'' y)^R                                   & & \text{by (B)} \\
&= (? \cdot x''y)^R                                               \\
&= (x''y)^R \cdot (?)^R                         & & \text{by $P(1)$}\\
&= \left( y^R \cdot (x'')^R \right) \cdot (?)^R & & \text{by $P(n)$} \\
&= y^R \cdot \left( (x'')^R \cdot (?)^R \right) & & \text{by ?}\\
&= y^R \cdot \left( ?x'' \right)^R              & & \text{by ?}\\
&= ?^R \cdot ?^R                                & & \text{by (B)} 
\end{align*}
Hence $P(n+1)$ holds.
QED.


{\bf Note.}
The intuition behind the proof is that 
given $x$ of length $n + 1$,
we cut it up into $x'$ and $x''$ of lengths 1 and $n$.
We then using the inductive hypothesis $P(1)$ and $P(n)$.
Read over the proof again and make sure the see the strategy
in the proof.
Writing proofs formally is one thing.
Understanding the strategy in a proof is another.
Only by studying lots of proofs and understanding their strategy
then will you really understand how to construct convincing proofs
of your own.

\newpage

Q5. Textbook Q 1.53.

{\bf SOLUTION.}

From $a \mid b$ we have:
\begin{align*}
a &\mid b  \\
\THEREFORE ax &= b \text{ for some integer $x$} & & \text{by ?} \tag{A}
\end{align*}
From $a \mid c$ we have:
\begin{align*}
a &\mid c  \\
\THEREFORE ay &= ? \text{ for some integer $?$} & & \text{by ?} \tag{B}
\end{align*}
From (A) and (B) we have:
\begin{align*}
           ax + ay &= b + ? \text{ for some integers $x, y$}               \\
\THEREFORE a?  &= b + ? \text{ for some integers $x, y$}     & & \text{by ?} \\
\THEREFORE a?  &= b + ? \text{ for some integer $x + y$}     & & \text{by ?} \\
\THEREFORE az  &= b + ? \text{ for some integers $z (= ?)$}  & & \text{by ?} \\
\THEREFORE ?   &\mid ?                                       & & \text{by ?} 
\end{align*}


\newpage


\newpage
WARNING!!! SPOILERS ON NEXT PAGE!!!

\newpage
HINT FOR 1.53:
In binary addition we have the following:
\begin{align*}
1 &= 0 + 1 \\
11 &= 0 + 11 \\
111 &= 0 + 111 \\
1111 &= 0 + 1111
\end{align*}


\end{document}
